%%%%%%%% ICML 2020 EXAMPLE LATEX SUBMISSION FILE %%%%%%%%%%%%%%%%%

\documentclass{article}

% Recommended, but optional, packages for figures and better typesetting:
\usepackage{microtype}
\usepackage{graphicx}
\usepackage{subfigure}
\usepackage{booktabs} % for professional tables
\usepackage{amsmath}
\usepackage{amssymb}
\usepackage{amsthm} % for proofs
% \allowdisplaybreaks

% hyperref makes hyperlinks in the resulting PDF.
% If your build breaks (sometimes temporarily if a hyperlink spans a page)
% please comment out the following usepackage line and replace
% \usepackage{icml2020} with \usepackage[nohyperref]{icml2020} above.
\usepackage{hyperref}

% Attempt to make hyperref and algorithmic work together better:
\newcommand{\theHalgorithm}{\arabic{algorithm}}

% Use the following line for the initial blind version submitted for review:
\usepackage{icml2020}

% If accepted, instead use the following line for the camera-ready submission:
%\usepackage[accepted]{icml2020}

% For notations
\def\~#1{\mathbb{#1}}
\def\*#1{\mathbf{#1}}
\newcommand{\mcal}[1]{\mathbf{\mathcal{#1}}}
\newcommand\norm[1]{\left\lVert#1\right\rVert}

\def\online{{\texttt{LineFilter}}}
\def\mr{{\texttt{MergeReduce}}}
\def\kernel{{\texttt{Kernel}}}
\def\kernelfilter{{\texttt{KernelFilter}}}
\def\mrlw{{\texttt{StreamingLW}}}
\def\mrwcb{{\texttt{StreamingWCB}}}
\def\mrfc{{\texttt{StreamingFC}}}
\def\uni{{\texttt{Uniform}}}
\def\stream{{\texttt{Streaming}}}

% For Theorem and Lemmas
\newtheorem{theorem}{Theorem}[section]
\newtheorem{corollary}{Corollary}[section]
\newtheorem{lemma}[theorem]{Lemma}

% The \icmltitle you define below is probably too long as a header.
% Therefore, a short form for the running title is supplied here:
\icmltitlerunning{Streaming Coresets for Symmetric Tensor Factorization}

\begin{document}

\twocolumn[
\icmltitle{Streaming Coresets for Symmetric Tensor Factorization}

% It is OKAY to include author information, even for blind
% submissions: the style file will automatically remove it for you
% unless you've provided the [accepted] option to the icml2020
% package.

% List of affiliations: The first argument should be a (short)
% identifier you will use later to specify author affiliations
% Academic affiliations should list Department, University, City, Region, Country
% Industry affiliations should list Company, City, Region, Country

% You can specify symbols, otherwise they are numbered in order.
% Ideally, you should not use this facility. Affiliations will be numbered
% in order of appearance and this is the preferred way.
\icmlsetsymbol{equal}{*}

\begin{icmlauthorlist}
\icmlauthor{Aeiau Zzzz}{equal,to}
\icmlauthor{Bauiu C.~Yyyy}{equal,to,goo}
\icmlauthor{Cieua Vvvvv}{goo}
\icmlauthor{Iaesut Saoeu}{ed}
\icmlauthor{Fiuea Rrrr}{to}
\icmlauthor{Tateu H.~Yasehe}{ed,to,goo}
\icmlauthor{Aaoeu Iasoh}{goo}
\icmlauthor{Buiui Eueu}{ed}
\icmlauthor{Aeuia Zzzz}{ed}
\icmlauthor{Bieea C.~Yyyy}{to,goo}
\icmlauthor{Teoau Xxxx}{ed}
\icmlauthor{Eee Pppp}{ed}
\end{icmlauthorlist}

\icmlaffiliation{to}{Department of Computation, University of Torontoland, Torontoland, Canada}
\icmlaffiliation{goo}{Googol ShallowMind, New London, Michigan, USA}
\icmlaffiliation{ed}{School of Computation, University of Edenborrow, Edenborrow, United Kingdom}

\icmlcorrespondingauthor{Cieua Vvvvv}{c.vvvvv@googol.com}
\icmlcorrespondingauthor{Eee Pppp}{ep@eden.co.uk}

% You may provide any keywords that you
% find helpful for describing your paper; these are used to populate
% the "keywords" metadata in the PDF but will not be shown in the document
\icmlkeywords{Tensor, Factorization, Subspace Embedding, Online, Streaming, Lp}

\vskip 0.3in
]

% this must go after the closing bracket ] following \twocolumn[ ...

% This command actually creates the footnote in the first column
% listing the affiliations and the copyright notice.
% The command takes one argument, which is text to display at the start of the footnote.
% The \icmlEqualContribution command is standard text for equal contribution.
% Remove it (just {}) if you do not need this facility.

%\printAffiliationsAndNotice{}  % leave blank if no need to mention equal contribution
\printAffiliationsAndNotice{\icmlEqualContribution} % otherwise use the standard text.

\begin{abstract}
Factorizing tensors has recently become an important optimization module in a number of machine learning pipelines, especially in latent variable models. We show how to do this efficiently in the streaming setting. Given a set of $n$ vectors, each in $\~R^d$, we present algorithms to select a sublinear number of these vectors as coreset, while guaranteeing that the CP decomposition of the $p$-moment tensor of the coreset approximates the corresponding decomposition of the $p$-moment tensor computed from the full data. We introduce two novel algorithmic techniques: online filtering and kernelization. Using these two, we present four algorithms that achieve different tradeoffs of coreset size, update time and working space, beating or matching various state of the art algorithms. In case of matrices (2-ordered tensor) our online row sampling algorithm guarantees $(1 \pm \epsilon)$ relative error spectral approximation. We show applications of our algorithms in learning single topic modeling. 
% Based on the application, from the input data one can create a $p-$order tensor ${\mcal T} = \sum_i \*a_{i}\otimes^{p}$, $p \ge 2$, and find its decomposed factors. Instead it can be found by factorizing the tensor created using vectors in $\*C$ only, and it has a guaranteed approximation (additive for odd $p$ and relative for even). 
% 
%Our online algorithm returns similar complexity as the state of the art offline sampling complexity.
%In case of matrices (2-ordered tensor) our online row sampling algorithm guarantees $(1 \pm \epsilon)$ relative error spectral approximation. %with $O(\frac{d\log d}{\epsilon^{2}}(1+2\log\|\*A\|))$ rows. 
% We demonstrate the performance of our algorithm empirically compared to other sampling strategies. 
% 
% We apply our method for single topic modeling for a document stream and streaming version of Gaussian mixture model to compare the performance of our algorithm with other sampling sampling strategies.
\end{abstract}

\section{Introduction}
Much of the data that is consumed in data mining and machine learning applications arrives in a streaming manner. The data is conventionally treated as a matrix, with a row representing a single data point and the columns its corresponding features. Since the matrix is typically large, it is advantageous to be able to store only a small number of rows and still preserve some of its ``useful" properties. One such abstract property that has proven useful in a number of different settings, such as solving regression, finding various factorizations, is {\em subspace preservation}. Given a matrix $\*A \in \~R^{n \times d}$ an $m \times d$ matrix $\*C$ is subspace preserving for the $\ell_2$ norm if, $\forall \*x \in \~R^{d}$,
\begin{align*}
    \big| \sum_{\tilde{\*a}_{j} \in \*C}(\tilde{\*a}_{j}^T \*x)^2 - \sum_{i \in [n]}(\*a_{i}^T \*x)^2\big|\leq\epsilon \cdot \sum_{i \in [n]}(\*a_{i}^T \*x)^2
\end{align*}
% where each $\mathbf{\hat{a}_i}$ is an appropriately scaled version of $\*a_{i}$. 
We typically desire $m \ll n$ and $\tilde{\*a}_{j}$'s represent the subsampled and rescaled rows from $\*A$. Such a sample $\*C$ is often referred to as a coreset.
This property has been used to obtain approximate solutions to many problems such as regression, low rank approximation etc.~\cite{woodruff2014sketching}, while having $m$ to be at most $O(d^2)$. Such property has been defined for other $\ell_p$ norms too \cite{dasgupta2009sampling,cohen2015p,clarkson2016fast}. 

Matrices are ubiquitous and depending on the application one can assume that the data is coming from a generative model, i.e. there is some distribution from which every incoming row is sampled and given to user. Many a times the goal is to know the hidden variables of this generative model. An obvious way to learn these variables is by representing data(matrix) by its low rank representation. However we know that a low rank representation of a matrix is not unique as there are various (such as SVD, QR, LRU) ways to decompose a matrix. So the hidden variables are not clear just by the low rank decomposition of the matrix form of the dataset.
% Instead for a dataset $\*A \in \~R^{n \times d}$ one can use a $p$ order tensor $\mcal T \~R^{d \times \ldots \times d}$ as $\mcal T = \sum_{i=1}^{n} \*a_{i} \otimes^{p}$, where $p$ is set by user depending the number of latent variables one is expecting in the generative model\cite{ma2016polynomial}. These are also called higher order moments of the dataset. The decomposition of such tensors are unique under a mild assumption. 
This is one of the reasons to look at higher order moments of the data i.e. tensors. Tensors are formed by outer product of data vectors, i.e. for a dataset $\*A \in \~R^{n \times d}$ one can use a $p$ order tensor $\mcal T \in \~R^{d \times \ldots \times d}$ as $\mcal T = \sum_{i=1}^{n} \*a_{i} \otimes^{p}$, where $p$ is set by user depending on the number of latent variables one is expecting in the generative model~\cite{ma2016polynomial}. The decomposition of such a tensor is unique under a mild assumption~\cite{kruskal1977three}. 
Factorization of tensors into its constituent elements has found uses in topic modeling~\cite{anandkumar2014tensor}, various latent variable models~\cite{anandkumar2012method,hsu2012spectral,jenatton2012latent}, training neural networks~\cite{janzamin2015beating} etc. 

For a $p$-order moment tensor $\mcal T = \sum_i \*a_{i}\otimes^p$ created using the set of vectors $\{\*a_{i}\}$ and for $\*x \in \~R^{d}$ one of the important property one needs to preserve is $\mcal T(\*x, _{\cdots}, \*x) = \sum_{i} (\*a_{i}^{T}\*x)^{p}$. This operation is also called tensor contraction \cite{song2016sublinear}. Now if we wish to ``approximate" it using only a subset of the vectors $\*A$, the above property for $\ell_2$ norm subspace preservation does not suffice. %What suffices, however, is a guarantee that is similar (not same) to that needed for the $\ell_p$ subspace preservation.  

For tensor factorization, which is performed using power iteration, a coreset $\*C \subset \bigcup_{i}\{\*a_{i}\}$, in order to give a guaranteed approximation to the tensor factorization, needs to satisfy the following natural extension of the $\ell_2$ subspace preservation condition:
\begin{align*}
\sum_{\*a_{i} \in \*A}({\*a_{i}}^T \*x)^p \approx \sum_{\tilde{\*a}_{j} \in \*C}({\tilde{\*a}_{j}}^T \*x)^p
\end{align*}
 Ensuring this tensor contraction property enables one to approximate the CP decomposition of $\mcal T$ using only the vectors $\tilde{\*a}_{i}$'s via power iteration method~\cite{anandkumar2014tensor}. A related notion is that of $\ell_p$ subspace embedding where we need that $\*C$ satisfies the following, $\forall \*x\in \~R^{d}$
\begin{align*}
\sum_{\*a_{i} \in \*A}|{\*a_{i}}^T \*x|^p \approx \sum_{\tilde{\*a}_{j} \in \*C}|{\tilde{\*a}_{j}}^T \*x|^p
\end{align*}
% This property ensures that we can approximate the $\ell_p$ regression problem by using only the rows in $\*C$. 
The two properties are the same for even $p$, as it is just sum of non negative terms. Due to the same reason they differ for odd values of $p$.

In this work, we show that it is possible to create coresets for the above property in streaming and restricted streaming ({\em online}) setting. In restricted streaming setting an incoming point, when it arrives, is either chosen in the set or discarded forever. We consider the following formalization of the above two properties. Given a query space of vectors $\*Q\subseteq \~R^{d}$ and $\epsilon > 0$, we aim to choose a set $\*C$ which contains sampled and rescaled rows from $\*A$ to ensure that $\forall \*x \in \*Q$ with probability at least $0.9$, the following properties hold,
\begin{align}
    \big|\sum_{\tilde{\*a}_{j} \in \*C}(\tilde{\*a}_{j}^T\*x)^p - \sum_{i \in [n]}(\*a_{i}^T \*x)^p\big| \leq \epsilon \cdot \sum_{i \in [n]}|\*a_{i}^T \*x|^p  \label{eq:contract} \\
    \big|\sum_{\tilde{\*a}_{j} \in \*C}|\tilde{\*a}_{j}^T\*x|^p - \sum_{i \in [n]}|\*a_{i}^T \*x|^p\big| \leq \epsilon \cdot \sum_{i \in [n]}|\*a_{i}^T \*x|^p  \label{eq:lp}
\end{align}
Note that neither property follows from the other. For even values of $p$, the above properties are identical and imply a relative error approximation as well. Where as for odd values of $p$, the $\ell_{p}$ subspace embedding as equation \eqref{eq:lp} gives a relative error approximation but the tensor contraction as equation \eqref{eq:contract} implies an additive error approximation, which becomes relative error under non-negativity constraints on $\*a_{i}$ and $\*x$. This happens, for instance, for the important use case of topic modeling, where $p=3$ typically.
% For even value of $p$ the above property ensures relative error approximation, where as for odd value of $p$ the above property give additive error approximation.
% 
\paragraph{Our Contributions:}
We give a method to sample rows in streaming manner for a $p$ order tensor, which is further decomposed to know the latent factors. For a given a matrix $\*A \in \~R^{n \times d}$, a $k$-dimensional query space $\*Q \in \~R^{k \times d}$, some integer $p \geq 2$ and $\epsilon > 0$,
\begin{itemize}
%    \item We show that in the offline setting, for any value of $p$, there exists a sample of appropriately reweighted rows of $\*A$ which satisfy equation \eqref{eqn1}. It can be achieved using well conditioned basis for $p$ norm. The size of sample is $\tilde{O}(\frac{d^{p}k}{\epsilon^{2}}\log(1/\delta))$ where $k$ is the dimension of the query space.
    \item We give an algorithm (\online) that is able to select rows, it takes $O(d^2)$ update time, and returns a sample of size $O(\frac{n^{1-2/p} k}{\epsilon^{2}} (d+d\log\|\*A\| - \min_{i}\log\|\*a_{i}\|))$ such that the set of selected rows form a coreset having the guarantees stated in equations ~\eqref{eq:contract} and \eqref{eq:lp} (Theorem~\ref{thm:Online}). It is streaming algorithm but also works well in the restricted streaming (online) setting.
% either a single row $\hat{\*a}_i$ (for $p$ even) or 
    \item We improve the sampling complexity of our coreset to $O(d^{p/2}k(\log n)^{10}\epsilon^{-5})$ by a streaming algorithm (\online+\mrlw) with amortized update time $O(d^2)$ (Theorem~\ref{thm:improvedStream-MR}). It requires slightly higher working space $O(d^{p/2}k(\log n)^{11}\epsilon^{-5})$.
    % 
    \item We present a kernelization technique which, for any vector $\*v$, creates two vectors $\grave{\*v}$ and $\acute{\*v}$ such that for any $\*x, \*y \in \~R^{d}$,
    $$|\*x^T \*y|^p = |\grave{\*x}^T \grave{\*y}|\cdot|\acute{\*x}^T \acute{\*y}|$$
    Using this technique, we give an algorithm (\kernelfilter) which takes $O(nd^{p+1})$ time and samples $O(\frac{k}{\epsilon^{2}} (d^{\lceil p/2 \rceil}+\lceil p/2 \rceil d^{\lceil p/2 \rceil}\log\|\*A\| - \lfloor p/2 \rfloor\min_{i}\log\|\*a_{i}\|))$ vectors to create a coreset having the same guarantee as~\eqref{eq:contract} and \eqref{eq:lp} (Theorem~\ref{thm:slowOnline}). The update time of the algorithm is $O(d^{p+1})$ and it requires an additional working space of $O(d^{p+1})$. It is streaming algorithm but also works well in the restricted streaming (online) setting.
    % 
    \item We combine both online algorithms and propose another algorithm (\online+\kernelfilter) which just takes $O(d^2)$ update time and returns $O(\frac{k}{\epsilon^{2}} (d^{\lceil p/2 \rceil}+\lceil p/2 \rceil d^{\lceil p/2 \rceil}\log\|\*A\| - \lfloor p/2 \rfloor\min_{i}\log\|\*a_{i}\|))$ vectors as coreset with same guarantees as equation~\eqref{eq:contract} and ~\eqref{eq:lp} (Theorem~\ref{thm:improvedOnlineCoreset}). This uses $O(d^{p+1})$ working space.
    % \item We give a communication complexity based lower bound (Theorem ~\ref{thm:lowerBound}) to prove that {\em any} one pass algorithm that creates a coreset having the same guarantee as~\eqref{eq:contract} or \eqref{eq:lp}, must use at least $\Omega(n^{1-2/p}/\log(n))$ rows.
    % 
    \item For the $p=2$ case, both \online~and \online+\kernelfilter~translate to an online algorithm for sampling rows of the matrix $\*A$, while guaranteeing a {\em relative error} spectral approximation (Theorem~\ref{thm:improvedMatrixCoreset}). This is an improvement (albeit marginal) over the online row sampling result by~\cite{cohen2016online}. The additional benefit of this new online algorithm over~\cite{cohen2016online} is that it does not need knowledge of $\sigma_{\min}(\*A)$ to give a relative error approximation. 
    % \item In case of matrix we also show that our sampling complexity is tight in terms of $O(d\log (\|\*A\|_{2})/\epsilon^{2})$.
\end{itemize}
The rest of this paper is organized as follows: In section \ref{sec:prelimnary}, we look at some preliminaries for tensors and coresets. We also describe the notation used throughout the paper. Section \ref{sec:related} discusses related work. In section \ref{sec:algorithms}, we state all the 4 streaming algorithms we propose. We also show how our problem of preserving tensor contraction relates to preserving $\ell_{p}$ subspace embedding. In section \ref{sec:proofs} we describe the guarantees given by our algorithm and some proofs. 
% The section ends with the lower bound and its proof. 
In section \ref{sec:topic}, we describe how our algorithm can be used in case of streaming single topic modeling. We also show some empirical results that compare our sampling scheme with other schemes. Proofs of most of the supporting lemmas have been delegated to the appendix~\ref{sec:appendix}, available in the supplementary material.
% The rest of this paper is organized as follows: In section 2, we look at some preliminaries for tensors and coresets. We also describe the notation used in the paper. Section 3 discusses related work. In section 4, we state all the 4 streaming algorithms we propose. We also summarizes the guarantees of these algorithms, whose supporting lemma and proofs are discussed in section 5. In section 6, we empirically show our algorithm can be used in streaming single topic modeling. Proofs of most of the supporting lemmas have been delegated to the appendix, available in the supplementary material.
% 
\begin{table*}[t]
\caption{Table Comparing Existing Work and Current Contributions.}
\label{tab:compare}
\vskip 0.1in
\begin{center}
\begin{small}
\begin{sc}
\begin{tabular}{|l|l|l|l|}
\hline
Algorithm & Sample Size $\tilde{O}(\cdot)$  & Update time   & Working space $\tilde{O}(\cdot)$\\
\hline\hline
\mrwcb~\cite{dasgupta2009sampling} & $d^{p}k\epsilon^{-2}$  & $d^{5}p\log d$ & $d^{p}k\epsilon^{-2}$ \\
\hline
\mrlw~\cite{cohen2015p} & $d^{p/2}k\epsilon^{-5}$  & $d^{p/2}$ & $d^{p/2}k\epsilon^{-5}$ \\
\hline
\mrfc~\cite{clarkson2016fast} & $d^{7p/2}\epsilon^{-2}$  & $d$ & $d^{7p/2}\epsilon^{-2}$ \\
\hline
\stream~\cite{dickens2018leveraging} & $n^{\gamma}d\epsilon^{-2}$  & $n^{\gamma}d^{5}$ & $n^{\gamma}d$ \\
\hline
\hline
\online~(Theorem~\ref{thm:Online}) & $n^{1-2/p}dk\epsilon^{-2}$ & $d^{2}$ & $d^2$ \\
\hline
\online+\mrlw~(Theorem~\ref{thm:improvedStream-MR}) & $d^{p/2}k\epsilon^{-5}$ & $d^{2}$ amortized & $d^{p/2}k\epsilon^{-5}$
%$O((1-2/p)^{10}d^{p/2}k(\log n)^{10}\epsilon^{-5})$
%$O((1-2/p)^{11}d^{p/2}k(\log n)^{11}\epsilon^{-5})$
\\
\hline
\kernelfilter~(Theorem~\ref{thm:slowOnline}) & $d^{\lceil p/2 \rceil}k\epsilon^{-2}$ & $d^{p}$ & $d^{p+1}$ \\
\hline
\online+\kernelfilter~(Theorem~\ref{thm:improvedOnlineCoreset}) & $d^{\lceil p/2 \rceil}k\epsilon^{-2}$ & $d^{2}$ amortized & $d^{p+1}$ \\
\hline 
\end{tabular}
\end{sc}
\end{small}
\end{center}
\vskip -0.1in
\end{table*}
\section{Preliminaries}{\label{sec:prelimnary}}
We use the following notation throughout the paper. A scalar is denoted by a lower case letter, e.g. $p$ while a vector is denoted by a boldface lower case letter, e.g. $\*a$. 
By default all vectors are considered as column vectors unless specified otherwise. Matrices and sets are denoted by boldface upper case letters, e.g. $\*A$. %The $i^{th}$ column and $j^{th}$ row of the matrix are denoted by $a_{i}$ and $a_{j}^{T}$ respectively.
Specifically, $\*A$ denotes an $n\times d$ matrix
with set of rows $\{\*a_{i}\}$ and, in the streaming setting, $\*A_{i}$ represents the matrix formed by the first $i$ rows of $\*A$ that have arrived. We will interchangeably refer to the set $\{\*a_i\}$ as the input set of vectors as well as the rows of the matrix $\*A$. A tensor is denoted by a bold calligraphy letter e.g. $\mcal T$. Given a set of $d-$dimensional vectors $\*a_1, \ldots, \*a_n$, from which a $p$-order symmetric tensor $\mcal T$ is obtained as
$ \mcal T = \sum_i^n \*a_i \otimes^{p}$ 
i.e. the sum of the $p$-th order outer product of each of the vectors. It is easy to see that a symmetric tensor $\mcal T$ satisfies
the following: $\forall i_{1},i_{2},_{\cdots},i_{p}; \mcal T_{i_{1},i_{2},_{\cdots},i_{p}} = \mcal T_{i_{2},i_{1},_{\cdots},i_{p}} = _{\cdots} = \mcal T_{i_{p},i_{p-1},_{\cdots},i_{1}}$, i.e. same value for all possible permutations of $(i_1, i_2, _{\cdots}, i_p)$. We define the scalar quantity, also known as tensor contraction, as $\mcal T(\*x,\ldots,\*x) = \sum_{i=1}^{n}(\*a_i^T\*x)^p$, where $\*x \in \~R^{d}$. There are three widely used tensor decomposition techniques known as CANDECOMP/PARAFAC(CP), Tucker, Hierarchical Tucker and Tensor Train decomposition. Our work focuses on CP decomposition. 
%In rest of the paper tensor decomposition is referred as CP decomposition.

We denote 2-norm for a vector $\*x$ as $\|\*x\|$, and any $p$-norm, for $p\neq 2$ as $\|\*x\|_p$. We denote the 2-norm or spectral norm of a matrix $\*A$ by $\|\*A\|$.
% 
\paragraph{Coresets.}
A coreset is a small summary of data which can give provable guarantees for a particular optimization problem. Formally, given a set
$\*X \subseteq \~R^d$, set of queries $\*Q$ and a nonnegative cost function $\mathnormal{f}_{\*q}(\*x)$ with parameter $\*q \in \*Q$ and data point $\*x \in \*X$, a set of subsampled and appropriately reweighted points $\*C$ is called a coreset if $\forall \*q \in \*Q$,  $|\sum_{\*x \in \*X}\mathnormal{f}_{\*q}(\*x) - \sum_{\tilde{\*x} \in \*C}\mathnormal{f}_{\*q}(\mathbf{\tilde{x}})| \leq \epsilon\sum_{\*x \in \*X}\mathnormal{f}_{\*q}(\*x)$ for some $\epsilon > 0$.
% We can relax the definition of coreset to allow a small additive error. For $\epsilon, \gamma > 0$, we can have a subset $\*C\subseteq \*X$ such that $\forall \*q \in \*Q$, $|\sum_{\*x \in \*X}\mathnormal{f}_{\*q}(\*x) - \sum_{\mathbf{\tilde{x}} \in \*C}\mathnormal{f}_{\*q}(\mathbf{\tilde{x}})| \leq \epsilon\sum_{\*x \in \*X}\mathnormal{f}_{\*q}(\*x) + \gamma$. 
% given a weighted dataset $\mathcal{X}$, let $\mathnormal{x} \in \mathcal{X}$ and $\mu_\mathcal{X}(\mathnormal{x})$ be its corresponding non-negative weight. 
% Let $\mathcal{Q}$ be a set of solutions known as query space and 
% $\mathnormal{q} \in \mathcal{Q}$ be a query and let 
% $\mathnormal{f}_\mathnormal{q}(\mathnormal{x})$ be a non-negative function. 

To guarantee the above approximation one can define a set of scores, termed as sensitivities \cite{langberg2010universal} corresponding to each point. This can be used to create coresets via importance sampling. The sensitivity of a point $ \*x $ is $s_{\*x} = \sup_{\*q \in \*Q} \frac{ \mathnormal{f}_{\*q}(\*x)}{\sum_{\*x' \in \*X} \mathnormal{f}_\*q(\*x')}$. In ~\cite{langberg2010universal} authors show that using any upper bounds to the sensitivity scores, we can create a probability distribution, which can be used to sample a coreset. The size of the coreset depends on the sum of these upper bounds and the dimension of the query space.
\section{Related Work}{\label{sec:related}}
Coresets are small summaries of data which 
can be used as a proxy to the original data with provable guarantees. The term was first introduced in \cite{agarwal2004approximating} where they used coresets for the shape fitting problem. Coresets for clustering problem were described in~\cite{har2004coresets}. 
In \cite{feldman2011unified} authors gave a generalized framework to construct coresets based on importance sampling using sensitivity scores introduced in \cite{langberg2010universal}. Interested reader can check \cite{woodruff2014sketching, braverman2016new, bachem2017practical}. Various online sampling schemes for spectral approximation are discussed in~\cite{cohen2016online, cohen2017input}. 

Tensor decomposition is unique under minimal assumptions \cite{kruskal1977three}. Therefore it has become very popular in various latent variable modeling applications \cite{anandkumar2012method, hsu2012spectral, anandkumar2014tensor}, neural networks \cite{janzamin2015beating} etc. However in general (i.e. without any assumption) most of the tensor problems including tensor decomposition are NP-hard \cite{hillar2013most}. There has been much work on fast tensor decomposition techniques. Various tensor sketching methods for tensor operations are discussed in \cite{bhojanapalli2015new, wang2015fast, song2016sublinear}. 
% They show that by applying FFT to the complete tensor during power iteration, one can save both time and space. This scheme can be used in combination with our scheme. 
The area of online tensor power iterations has also been explored in \cite{huang2015online, wang2016online}. 
% A work on element wise tensor sampling \cite{bhojanapalli2015new} gives a distribution on all the tensor elements and samples few entries accordingly. For $3^{rd}$ order, orthogonally decomposable tensors, \cite{song2016sublinear} gives a sub-linear time algorithm for tensor decomposition which requires the knowledge of norms of slices of the tensor. 
Various heuristics for tensor sketching as well as RandNLA techniques \cite{woodruff2014sketching} over matricized tensors for estimating low rank tensor approximation have been studied in \cite{song2019relative}.

In the online setting, for a matrix $\*A \in \~R^{n \times d}$ where rows are coming in streaming manner, the guarantee achieved by \cite{cohen2016online} while preserving additive error spectral approximation with sample size $O(d(\log d)(\log \epsilon\|\*A\|^{2}/\delta))$. $|\|\*A\*x\|^{2} - \|\*C\*x\|^{2}| \leq \epsilon \|\*A\*x\|^{2} + \delta, \forall \*x \in \~R^{d}$.

The problem of $\ell_{p}$ subspace embedding has been explored in both offline~\cite{dasgupta2009sampling, woodruff2013subspace, cohen2015p, clarkson2016fast} and streaming setting~\cite{dickens2018leveraging}. As any offline algorithm can be used as streaming algorithm~\cite{har2004coresets}, we use the known offline algorithms and summarize their results in streaming version in table \ref{tab:compare}.
% In \cite{dasgupta2009sampling} the authors show that one can spend $O(nd^{5}\log n)$ time to sample $O(\frac{d^{p+1}}{\epsilon^{2}})$ rows to get a guaranteed $(1\pm \epsilon)$ approximate subspace embedding for any $p$. 
The algorithm in~\cite{woodruff2013subspace} samples $O(n^{1-2/p}\mbox{poly}(d))$ rows and gives $\mbox{poly}(d)$ error relative subspace embedding but in $O(\mbox{nnz}(A))$ time. For streaming $\ell_{p}$ subspace embedding~\cite{dickens2018leveraging}, give a one pass deterministic algorithm for $\ell_{p}$ subspace embedding for $1\leq p\leq \infty$. 
%Note that our lower bound result shows that getting the number of row samples to be only a function of $d$ is not possible for $(1 \pm \epsilon)$ error in the one pass streaming / online setting. 
% As they also use well conditioned basis to decide sampling probability for the rows, hence it can also be used to preserve tensor contraction.
For some constant $\gamma \in (0,1)$ the algorithm takes $O(n^{\gamma}d)$ space and $O(n^{\gamma}d^{2}+n^{\gamma}d^{5}\log n)$ update time to return a $1/d^{O(1/\gamma)}$ error relative subspace embedding for any $\ell_{p}$ norm. 
% This, however, cannot be made into a constant factor approximation. We propose an online randomized algorithm that gives a guaranteed $(1\pm\epsilon)$ relative error approximation. 
%\section{Algorithms for $\ell_p$ Subspace Preservation}
\section{Algorithms and Guarantees}
{\label{sec:algorithms}}
In this section we discuss our two major contributions. We first introduce the two algorithmic modules--\online~and \kernelfilter. \online, on arrival of each row, simply decides whether to sample it or not. The probability of sampling is computed based on the stream seen till now,
% \kernelfilter, given a row $\*a_i$, either creates a single row $\hat{\*a}$ if $p$ is even, or creates two rows $\grave{\*a}_{i}$ and $\acute{\*a}_{i}$ if $p$ is odd such that the following holds: for any vector $\*x$, there is a similar transformation of $\*x$ either to $\hat{\*x}$ (for even $p$), or to a pair $(\grave{\*x}, \acute{\*x})$ such that,
% \begin{align*}
%       (\*a_{i}^{T}\*x)^{p} =
%   \begin{cases}
%     (\hat{\*a}_{i}^{T}\hat{\*x})^{2}  & \quad \text{if } p\ \mbox{ even},\\
%     (\grave{\*a}_{i}^{T}\grave{\*x})(\acute{\*a}_{i}^{T}\acute{\*x})  & \quad \text{if } p\ \mbox{odd}.\\
%   \end{cases}
% \end{align*}
where as in \kernelfilter, for every incoming row $\*a_i$, the decision of sampling it, depends on two rows $\grave{\*a}_{i}$ and $\acute{\*a}_{i}$ we create from $\*a_{i}$
such that: for any vector $\*x$, there is a similar transformation $(\grave{\*x}$ and $\acute{\*x})$ and we get, $|\*a_{i}^{T}\*x|^{p} = |\grave{\*a}_{i}^{T}\grave{\*x}|\cdot|\acute{\*a}_{i}^{T}\acute{\*x}|$. We call it kernelization.
% \begin{align*}
%       (\*a_{i}^{T}\*x)^{p} =
%   \begin{cases}
%     (\hat{\*a}_{i}^{T}\hat{\*x})^{2}  & \quad \text{if } p\ \mbox{ even},\\
%     (\grave{\*a}_{i}^{T}\grave{\*x})(\acute{\*a}_{i}^{T}\acute{\*x})  & \quad \text{if } p\ \mbox{odd}.\\
%   \end{cases}
% \end{align*}

Note that both \online~and \kernelfilter~are restricted streaming algorithms in the sense that each row is selected / processed only when it arrives. This online nature of the two modules allows us to use these as modules in order to create the following algorithms 
\begin{enumerate}
    \item \online+\mrlw, in which the output streams of \online~is fed to a \mrlw, which is a merge-and-reduce based streaming algorithm based on Lewis Weights. Here \mrlw~outputs the final coreset. 
    % \item \online+\kernelfilter: The output of \online~is first kernelized into either a single row (or two rows). These rows are then sampled using, what is essentially a version of \online~for $p=2$ case. 
    \item \online+\kernelfilter: The output of \online~is first kernelized into two rows. These rows are then sampled using, what is essentially a version of \online~for $p=2$ case. 
\end{enumerate}
The algorithms compute a score for every incoming row and based on the score the sampling probability of the row is decided. The score depends on the incoming row (say $\*x_{i}$) and some prior knowledge (say $\*M$) of the data which we have already seen. Here, we define $\*M = \*X_{i-1}^{T}\*X_{i-1}$ and $\*Q$ is its orthonormal column basis where $\*X_{i-1}$ represents the matrix with rows $\*x_{1},_{\ldots},\*x_{i-1}$. Now we present the method which is called by both \online~and \kernelfilter~for computing sampling probability.
% 
\begin{algorithm}[htpb]
\caption{OnlineScore($\*x_{i}, \*M, \*Q, p, r$)}{\label{alg:onineScore}}
\begin{algorithmic}
\IF{$\*x_{i} \in \mbox{column space}(\*Q)$}
\STATE $\*M_{I} = \*M^{\dagger} - \frac{\*M^{\dagger}\*x_{i}\*x_{i}^{T}\*M^{\dagger}}{1+\*x_{i}^{T}\*M^{\dagger}\*x_{i}}$ 
\STATE $\*M = \*M + (\*x_{i}\*x_{i}^T)$ \;
\ELSE
\STATE $\*M = \*M + \*x_{i}\*x_{i}^{T}$; $\*M_{I} = \*M^{\dagger}$
\STATE $\*Q = \mbox{orthonormal column basis}(\*M)$
\ENDIF
\STATE $\tilde{e}_{i} = \*x_{i}^T\*M_{I}\*x_{i}$
\STATE Return $\tilde{e}_{i},\*M,\*Q$
\end{algorithmic}
\end{algorithm}

Here if the incoming row $\*x_{i}$ lies in the subspace spanned by $\*Q$ (i.e. if $\|\*Q\*x_{i}\|=\|\*x_{i}\|$), then the algorithm take $O(m^{2})$ time else it takes $O(m^{3})$ where $\*x_{i} \in \~R^{m}$. Here we have used a modified version of Sherman Morrison formula to compute $(\*X_{i}^{T}\*X_{i})^{\dagger} = (\*X_{i-1}^{T}\*X_{i-1}+\*x_{i}\*x_{i}^{T})^{\dagger} = (\*M+\*x_{i}\*x_{i}^{T})^{\dagger}$. Note that in our setup $\*M$ need not be full rank, so we use the formula $(\*X_{i}\*X_{i})^{\dagger} = \*M^{\dagger} - \frac{\*M^{\dagger}\*x_{i}\*x_{i}^{T}\*M^{\dagger}}{1+\*x_{i}^{T}\*M^{\dagger}\*x_{i}}$. We prove this formula as a lemma in the appendix~\ref{lemma:modified-SM}. 
% In this section, we give an improved streaming algorithm and two online algorithms. But before discuss our result,
% first we show that the our coresets for tensor contraction
% also preserves $\ell_p$ subspace embedding algorithm in~\cite{dasgupta2009sampling}. 
% For simplicity we show this relation between in the offline setting.
% For a matrix $\*A \in \~R^{n \times d}$, we intend to preserve the property in eqn \eqref{eq:contract}. We create $\*C$ by sampling original vectors $\*a_{i}$ with appropriate scaling. In order to reduce the variance of the difference between the original and sampled quantity we use sensitivity based framework to decide our sampling probability. Sensitivity scores are well defined for positive cost function. Now in our problem for odd $p$ and for some $\*x$ the cost $(\*a_{i}^{T}\*x)^{p}$ could be negative. for row $i$ we define sensitivity score as,
% \begin{align*}
%  s_{i} = \sup_{\*x}\frac{|\*a_{i}^{T}\*x|^{p}}{\sum_{j=1}^{n}|\mathbf{a_{j}}^{T}\*x|^{p}}
% \end{align*}
% Note that by sampling enough number of rows based on above sensitivity scores also preserves $\sum_{i=1}^{n}|\*a_{i}^{T}\*x|^{p} = \|\*A\*x\|_{p}^{p}$ \cite{langberg2010universal}. get a coreset $\*C$ which is $\ell_{p}$ subspace embedding, i.e. $\forall \*x$, $|\|\*A\*x\|_{p}^{P} - \|\*C\*x\|_{p}^{p}| \leq \epsilon \|\*A\*x\|_{p}^{p}$. With appropriate scaling the method in \cite{dasgupta2009sampling} also achieves the guarantee in equation (\ref{eq:lp}) for any $\ell_{p}$. We know that any offline algorithm can be converted into a streaming algorithm using merge and reduce method \cite{har2004coresets}.
% Below We summarize the guarantees of streaming $\ell_{p}$ subspace embedding. 
% \begin{lemma}\label{lemma:Stream-MR}
%  Given a set of $n$ streaming vectors $\{\*a_{i}\}$, and $\*Q$, a fixed $k$-dimensional subspace, then the algorithm can be used using merge-and-reduce method which returns $\*C$ such that, with probability
% $1 - \delta$ and $\epsilon > 0$, $\forall \*x\in \*Q$, we have that 
%  \begin{align}{\label{eqn:stream-MR}}
%   \big|\sum_{\mathbf{\hat{a}_j}\in \*C}(\mathbf{\hat{a}_{j}}^T\*x)^{p}-\sum_{i=1}^{n}(\*a_{i}^T\*x)^{p}\big| \leq \epsilon\sum_{i=1}^{n}|\*a_{i}^T\*x|^{p}
%  \end{align}
%  The algorithm requires $O(d^{5}(p\log d - 2\log \epsilon))$ amortized update time and $O(d^{p+1}k\epsilon^{-2}(\log n)^{7})$ working space to return $\*C$ of size $O(\frac{d^{p}k(\log n)^{6}}{\epsilon^{2}}\log(1/\delta))$.
% \end{lemma}
% The proof is fairly straight forward and we discuss it in the appendix.
% 
% Now we discuss the first online algorithm and give its guarantees. The time complexity of this algorithm is as low as $nd^{2}$ but it returns a coreset which is only $o(n)$. Next we utilize our online algorithm and the fact that any offline algorithm can be converted into a streaming algorithm using merge and reduce method \cite{har2004coresets}, we propose a faster streaming algorithm. We feed the output of our algorithm to merge and reduce method. In this case we use the $\ell_{p}$ sampling method \cite{dasgupta2009sampling} and we get a smaller coreset at a cost of slightly higher working space. Note that this entire algorithm will have a dominated update time for each row same as update time of our fast online algorithm. Next we show that a $p$ order tensor contraction can be reduced 2-order tensor or matrices. This online algorithm takes more time but returns smaller coreset which is independent of $n$. Finally we give an algorithm which takes benefit of both our online methods and give an improved time and space complexity online algorithm. 
% % 
\subsection{\online}
Here we present our first streaming algorithm which ensures equations~\eqref{eq:contract} and~\eqref{eq:lp}. The algorithm can also be used in restricted steaming (online) setting where for every incoming row we get only one chance to decide whether to sample it or not. Due to its nature of filtering out rows we call it \online~algorithm. The algorithm tries to reduce the variance of the difference between the original and the sampled term. In order to achieve that we use sensitivity based framework to decide the sampling probability of each row. The sampling probability of a row is proportional to its sensitivity scores. In some sense the sensitivity score of a row represents how much the variance of the difference is going to affect if that row is not present in the sample. We discuss it in detail in section \ref{sec:proofs}.
Now we present the \online~algorithm and its corresponding guarantees. 
\begin{algorithm}[htpb]
\caption{\online~}{\label{alg:onlineCoreset}}
\begin{algorithmic}
\REQUIRE Streaming rows $\*a_{i}^T, i = 1, {}_{\cdots} n, p \geq 2, r > 1$
\ENSURE Coreset $\*C$ satisfying eqn \eqref{eq:contract} and \eqref{eq:lp} w.h.p.
\STATE $\*M = \*0^{d \times d}$, $L=0$, $\*C= \emptyset$
\STATE $\*Q =  \mbox{orthonormal column basis of }\*M$
\WHILE{current row $\*a_{i}^T$ is not the last row}
\STATE $[\tilde{e}_{i}, \*M, \*Q]$ = OnlineScore($\*a_{i},\*M,\*Q,p,r$)
% \IF{$\*a_{i} \in \mbox{row space}(\mathbf{Q})$}
% \STATE $\mathbf{M}_{I} = \mathbf{M}_{2}^{\dagger} - \frac{\mathbf{M}_{2}^{\dagger}\*a_{i}\*a_{i}^{T}\mathbf{M}_{2}^{\dagger}}{1+\*a_{i}^{T}\mathbf{M}_{2}^{\dagger}\*a_{i}}$ 
% \STATE $\mathbf{M_{2}} = \mathbf{M_{2}} + (\*a_{i}\*a_{i}^T)$ \;
% \ELSE
% \STATE $\*M = \*M + \*a_{i}\*a_{i}^{T}$; $\*M_{I} = \*M^{\dagger}$
% \STATE $\*Q = \mbox{orthonormal row basis}(\*M)$
% \ENDIF
\STATE $\tilde{l}_{i} = \min\{i^{p/2-1}(\tilde{e}_{i})^{p/2},1\}$
\STATE $L = L+\tilde{l}_{i}; p_{i} = \min\{r\tilde{l}_{i}/L,1\}$
\STATE Sample $\*a_{i}/\sqrt[p]{p_{i}}$ in $\*C$ with probability $p_{i}$
\ENDWHILE
\STATE Return $\*C$
\end{algorithmic}
\end{algorithm}

Every time a row $\*a_{i} \in \~R^{d}$ comes, the \online~ calls the function~\ref{alg:onineScore} (i.e. $OnlineScore()$) which returns a score $\tilde{e}_{i}$. Then \online~computes $\tilde{l}_{i}$, which is an upper bound of the sensitivity score. Based on $\tilde{l}_{i}$ the row's sampling probability is decided. We formally define and discuss sensitivity score of our problem in section \ref{sec:proofs}.
Now for the $OnlineScore()$ function there can be at most $d$ occasions where an incoming row is not in the row space of the previously seen rows. In these cases the function takes $O(d^{3})$ time and for the other at least $n-d$ cases it takes $O(d^{2})$ time to return $\tilde{e}_{i}$. Hence the entire algorithm \online~ takes just $O(nd^{2})$. Now we summarize the guarantees of the \online~in the following theorem.
\begin{theorem}\label{thm:Online}
Given $\*A \in \~R^{n \times d}$ whose rows are coming in streaming manner, \online~selects a set $\*C$ of size $O(\frac{n^{1-2/p}k}{\epsilon^{2}}(d+d\log\|\*A\|-\min_{i} \log \|\*a_{i}\|))$ using both working space and update time $O(d^2)$. Suppose $\*Q$ is a fixed $k$-dimensional subspace, then with probability at least 0.9, for $\epsilon > 0$, $\forall \*x \in \*Q$, the set $\*C$ satisfies both tensor contraction and $\ell_{p}$ subspace embedding as in equation \eqref{eq:contract} and \eqref{eq:lp} respectively.
% \begin{align}{\label{eqn:onlineGuarantee}}
%  \big|\sum_{\mathbf{\hat{a}_j} \in \*C}(\mathbf{\hat{a}_{j}}^T\*x)^{p}-\sum_{i=1}^{n}(\*a_{i}^T\*x)^{p}\big| \leq \epsilon\sum_{i=1}^{n}|\*a_{i}^T\*x|^{p}
% \end{align}
\end{theorem}
It is worth noting that \online~ benefits by taking very less working space and computation time, which are independent of $p$ i.e. order of the tensor. However \online~gives a coreset which is sublinear to input size, which is not completely independent of input size $n$.
% 
\subsection{\online+\mrlw}
Here we present a streaming algorithm which returns a coreset for the same problem with its coreset size independent of $n$. First we want to point out that our coresets for tensor contraction i.e. equation \eqref{eq:contract} also preserves $\ell_p$ subspace embedding i.e. equation \eqref{eq:lp},
For simplicity we show this relation in a complete offline setting, where we have access to the entire data $\*A$.
For a matrix $\*A \in \~R^{n \times d}$, we intend to preserve the tensor contraction property in equation \eqref{eq:contract}. We create $\*C$ by sampling original row vectors $\*a_{i}$ with appropriate scaling. 
We analyze the variance of the difference between the original and sampled term, through Bernstein inequality ~\cite{dubhashi2009concentration} and try to reduce it. 
Here we use sensitivity based framework to decide our sampling probability where we know sensitivity scores are well defined for positive cost function~\cite{langberg2010universal}. Now with the tensor contraction problem for odd $p$ and for some $\*x$, the cost $(\*a_{i}^{T}\*x)^{p}$ could be negative. So for every row $i$ we define the sensitivity score as follows,
\begin{equation}{\label{eqn:sensitivity}}
 s_{i} = \sup_{\*x}\frac{|\*a_{i}^{T}\*x|^{p}}{\sum_{j=1}^{n}|\mathbf{a_{j}}^{T}\*x|^{p}}
\end{equation}
Here by sampling enough number of rows based on above defined sensitivity scores also preserve $\sum_{i=1}^{n}|\*a_{i}^{T}\*x|^{p} = \|\*A\*x\|_{p}^{p}$ \cite{langberg2010universal}. The sampled rows creates a coreset $\*C$ which is $\ell_{p}$ subspace embedding, i.e. $\forall \*x$, $|\|\*A\*x\|_{p}^{P} - \|\*C\*x\|_{p}^{p}| \leq \epsilon \|\*A\*x\|_{p}^{p}$. We define the online version of these scores in section~\ref{sec:proofs} which also preserves tensor contraction. It is not difficult to show that the offline scores defined above also preserves tensor contraction.
Sampling based methods used in~\cite{dasgupta2009sampling, cohen2015p, clarkson2016fast} to get a coreset for $\ell_{p}$ subspace embedding also preserve tensor contraction. This is because all these sampling based methods try reducing the variance of the difference between original and expected term.
% Now with similar argument as the algorithms in \cite{dasgupta2009sampling, cohen2015p, clarkson2016fast} uses sampling based method and returns a coreset which guarantees $\ell_{p}$ subspace embedding by reducing the variance of the difference between original and expected term, hence the same coreset can also be used to guarantee tensor contraction~\eqref{eq:contract}.

As We know that any offline algorithm can be made a streaming algorithm using merge and reduce method~\cite{har2004coresets}. For $p \geq 2$ the sampling complexity of \cite{cohen2015p} is best among all other methods we mentioned. Hence here we use Lewis Weights sampling \cite{cohen2015p} as the offline method along with merge and reduce for a streaming algorithm which we call \mrlw. The following lemma summarizes the guarantee one gets from \mrlw.
\begin{lemma}\label{lemma:Stream-MR}
 Given a set of $n$ streaming vectors $\{\*a_{1},_{\cdots}, \*a_{n}\}$, and $\*Q$, a fixed $k$-dimensional subspace, then \mrlw~returns $\*C$ such that, with probability 0.9 and $\epsilon > 0$, $\forall \*x\in \~R^{d}$, it satisfies tensor contraction and $\ell_{p}$ subspace embedding as in equation \eqref{eq:contract} and \eqref{eq:lp} respectively.

%  \begin{align}{\label{eqn:stream-MR}}
%   \big|\sum_{\mathbf{\hat{a}_j}\in \*C}(\mathbf{\hat{a}_{j}}^T\*x)^{p}-\sum_{i=1}^{n}(\*a_{i}^T\*x)^{p}\big| \leq \epsilon\sum_{i=1}^{n}|\*a_{i}^T\*x|^{p}
%  \end{align}
 It requires $O(d^{p/2})$ amortized update time and uses $O(d^{p/2}\epsilon^{-5}\log^{11} n)$ working space to return a coreset $\*C$ of size $O(d^{p/2}\epsilon^{-5}\log^{10} n)$.
\end{lemma}
The proof is fairly straight forward which we discuss in the appendix~\ref{proof:Stream-MR}. Note that in this case both update time and working space are functions of $d$ and $p$. 

Now we propose our second algorithm where we feed the output of \online~to \mrlw~method. Here every incoming row is fed to \online, which quickly computes a sampling probability and based on which the row gets sampled. Now if it gets sampled then we pass it to the \mrlw~method, which returns a coreset. The entire algorithm gets an improved {\em amortized} update time compared to \mrlw~and improved sampling complexity compared to \online. We call this algorithm \online+\mrlw~and summarize the guarantees in the following theorem.
\begin{theorem}{\label{thm:improvedStream-MR}}
 Consider $\*A \in \~R^{n \times d}$ whose rows are given to \online+\mrlw~in streaming manner. It requires $O(d^{2})$ amortized update time and uses $((1-2/p)^{11}d^{p/2}\epsilon^{-5}\log^{11} n)$ working space to return a coreset $\*C$ of size $((1-2/p)^{10}d^{p/2}\epsilon^{-5}\log^{10} n)$ such that with at least 0.9 probability, $\*C$ satisfies both tensor contraction and $\ell_{p}$ subspace embedding as in equation \eqref{eq:contract} and \eqref{eq:lp} respectively.
%  \begin{align}{\label{eqn:improvedStream-MR}}
%   \big|\sum_{\mathbf{\hat{a}}_j\in \*C}(\mathbf{\hat{a}_{j}}^T\*x)^{p}-\sum_{i=1}^{n}(\*a_{i}^T\*x)^{p}\big| \leq \epsilon\sum_{i=1}^{n}|\*a_{i}^T\*x|^{p}
%  \end{align}
\end{theorem}
% 
This is an improved streaming algorithm which gives the same guarantee as lemma \ref{lemma:Stream-MR} but using very less amortized update time. Hence asymptotically we get an improvement in the overall run time of the algorithm and yet get a coreset which is independent of $n$.
It is important to note that we could improve the run time of the streaming result because our \online~can be used in online setting, which returns a sub-linear size coreset (i.e. $o(n)$) and its update time is less than the amortized update time of \mrlw. The proof of the above theorem is very simple and discussed in the appendix~\ref{thm:improvedStream-MR}. Note that \online+\mrlw~is a streaming algorithm, whereas \online~or the next algorithm that we propose, works even in restricted streaming setting. 
% 
\subsection{\kernelfilter}
Now we discuss our second streaming algorithm for the tensor contraction guarantee as equation \eqref{eq:contract}. First we give a reduction from $p$-order to $2$-order tensor contraction.
\begin{lemma}{\label{lemma:kernel}}
 For a vector $\*x \in \~R^{d}$ it can be transformed to $(\grave{\*x}$ and $\acute{\*x})$ such that for any two d-dimensional vectors $\*x$ and $\*y$ with their similar transformations we get,
 $$|\*x^{T}\*y|^{p} = |\grave{\*x}^{T}\grave{\*y}|\cdot|\acute{\*x}^{T}\acute{\*y}|$$
\end{lemma}
% 
%  \[|\*x^{T}\*y|^{p} =
%   \begin{cases}
%     |\hat{\*x}^{T}\hat{\*y}|^{2}  & \quad \text{if } p \mbox{ even}\\
%     |\grave{\*x}^{T}\grave{\*y}||\acute{\*x}^{T}\acute{\*y}|  & \quad \text{if } p\ \mbox{odd}\\
%   \end{cases}
%  \]
% % 
The proof is simple which we discuss in the appendix~\ref{proof:kernel}. Specifically for even valued $p$ , $(\grave{\*x},\grave{\*y})$ are same as $(\acute{\*x},\acute{\*y})$. So for simplicity in future reference we use $|\*x^{T}\*y|^{p} = |\hat{\*x}^{T}\hat{\*y}|^{2}$ where we refer $\hat{\*x} = \grave{\*x} = \acute{\*x}$ and similarly $\hat{\*y} = \grave{\*y} = \acute{\*y}$ for even valued $p$.

% Here a term $|\*a_{i}^{T}\*x|^{p}$ is defined as $|\*a_{i}^{T}\*x|^{\lfloor p/2 \rfloor}|\*a_{i}^{T}\*x|^{\lceil p/2 \rceil}$. Here write $|\*a_{i}^{T}\*x|^{\lfloor p/2 \rfloor} = |\grave{\*a}_{i}^{T}\grave{\*x}|$ and $|\*a_{i}^{T}\*x|^{\lceil p/2 \rceil} = |\acute{\*a}_{i}^{T}\acute{\*x}|$. For even valued $p$ it is $|\*a_{i}^{T}\*x|^{\lfloor p/2 \rfloor}=|\*a_{i}^{T}\*x|^{\lceil p/2 \rceil} = |\*a_{i}^{T}\*x|^{p/2}$. Where we represent $|\*a_{i}^{T}\*x|^{p} = |\langle \*a \otimes^{p/2}, \*x \otimes^{p/2} \rangle|^{2} = |\hat{\*a}_{i}^{T}\hat{\*x}|^{2}$. Here the vector $\hat{\*a}_{i} = \mbox{vec}(\*a_{i} \otimes^{p/2}) \in \~R^{p/2}$ and similarly $\hat{\*x}$ is also defined. Now for odd value of $p$ we have $\grave{\*a}_{i} = \mbox{vec}(\*a_{i} \otimes^{(p-1)/2}) \in \~R^{(p-1)/2}$ and $\acute{\*a}_{i} = \mbox{vec}(\*a_{i} \otimes^{(p+1)/2}) \in \~R^{(p+1)/2}$. Similarly $\grave{\*x}$ and $\acute{\*x}$ is defined for odd value of $p$.
% \[\sum_{i=1}^{n} |\*a_{i}^{T}\*x|^{p} =
%   \begin{cases}
%   \sum_{i=1}^{n} |\hat{\*a}_{i}^{T}\hat{\*x}|^{2}  & \quad \text{if } p \mbox{ even}\\
%   \sum_{i=1}^{n} |\grave{\*a}_{i}^{T}\grave{\*x}||\acute{\*a}_{i}^{T}\acute{\*x}|  & \quad \text{if } p\ \mbox{odd}\\
%   \end{cases}
% \]
Now we give a streaming algorithm which is in the same spirit of \online, which computes the sampling probability based on $\grave{\*a}_{i}, \acute{\*a}_{i}$ and the counterparts of the previously seen rows. Since the vectors $\grave{\*a}_{i}$ and $\acute{\*a}_{i}$ only depend on $\*a_{i}$, this algorithm can also be used in online setting. 
% Now for every incoming row based on the value of $p$ our algorithm converts the $d$ dimensional vector into its corresponding higher dimensional vectors before deciding its sampling complexity. 
Since we give a sampling based coreset, it retains the structure of the input data. So one need not convert $\*x$ into its corresponding higher dimensional vector, instead one can use the same $\*x$ on the sampled coreset to compute the desired operation. We call it \kernelfilter~and give it as algorithm~\ref{alg:slowOnline}.
% 
\begin{algorithm}[htpb]
\caption{\kernelfilter}{\label{alg:slowOnline}}
\begin{algorithmic}
\REQUIRE Streaming rows $\*a_{1}, \*a_{2}, _{\cdots}, \*a_{n}$, $r>1, p\geq2$
\ENSURE Coreset $\*C$ satisfying eqn \eqref{eq:contract} and \eqref{eq:lp} w.h.p.
% \IF{$p$ is odd}
\STATE $\grave{\*M} = 0^{d^{\lfloor p/2 \rfloor} \times d^{\lfloor p/2 \rfloor}};\acute{\*M} = 0^{d^{\lceil p/2 \rceil} \times d^{\lceil p/2 \rceil}}$
\STATE $L=0, \*C= \emptyset$
\STATE $\grave{Q} = \mbox{orthonormal column basis of }\grave{\*M}$
\STATE $\acute{Q} = \mbox{orthonormal column basis of }\acute{\*M}$
\WHILE {$i \leq n$}
\STATE $\grave{\*a}_{i} = \mbox{vec}(\*a_{i}\otimes^{\lfloor p/2 \rfloor}); \acute{\*a}_{i} = \mbox{vec}(\*a_{i}\otimes^{\lceil p/2 \rceil})$
\STATE $[\grave{e}_{i}, \grave{\*M}, \grave{\*Q}]$ = OnlineScore($\grave{\*a}_{i},\grave{\*M},\grave{\*Q},p,r$)
\STATE $[\acute{e}_{i}, \acute{\*M}, \acute{\*Q}]$ = OnlineScore($\acute{\*a}_{i},\acute{\*M},\acute{\*Q},p,r$)
\STATE $\tilde{l}_{i} = (\grave{e}_{i})^{1/2}; \acute{l}_{i} = (\acute{e}_{i})^{1/2}$
\STATE $\tilde{l}_{i} = \grave{l}_{i}\acute{l}_{i}; L = L+\tilde{l}_{i}; p_{i} = \min\{r\tilde{l}_{i}/L,1\}$
\STATE Sample $\*a_{i}/\sqrt[p]{p_{i}}$ in $\*C$ with probability $p_{i}$
% \IF{$\grave{\*a}_{i} \in \mbox{row space}(\grave{Q})$}
% \STATE $\grave{\*M}_{I} = \grave{\*M}^{\dagger} - \frac{\grave{\*M}^{\dagger}\grave{\*a}_{i}\grave{\*a}_{i}^{T}\grave{\*M}^{\dagger}}{1+\grave{\*a}_{i}^{T}\grave{\*M}^{\dagger}\grave{\*a}_{i}}; \grave{\*M} = \grave{\*M} + \grave{\*a}_{i}\grave{\*a}_{i}^{T}$
% \ELSE
% \STATE $\grave{\*M} = \grave{\*M} + \grave{\*a}_{i}\grave{\*a}_{i}^{T}; \grave{\*M}_{I} = \grave{\*M}^{\dagger}$
% \STATE $\grave{\*Q} = \mbox{orthonormal row basis}(\grave{\*M})$
% \ENDIF
% \IF{$\acute{\*a}_{i} \in \mbox{row space}(\acute{Q})$}
% \STATE $\acute{\*M}_{I} = \acute{\*M}^{\dagger} - \frac{\acute{\*M}^{\dagger}\acute{\*a}_{i}\acute{\*a}_{i}^{T}\acute{\*M}^{\dagger}}{1+\acute{\*a}_{i}^{T}\acute{\*M}^{\dagger}\acute{\*a}_{i}}; \acute{\*M} = \acute{\*M} + \acute{\*a}_{i}\acute{\*a}_{i}^{T}$
% \ELSE
% \STATE $\acute{\*M} = \acute{\*M} + \acute{\*a}_{i}\acute{\*a}_{i}^{T}$; $\acute{\*M}_{I} = \acute{\*M}^{\dagger}$
% \STATE $\acute{\*Q} = \mbox{orthonormal row basis}(\acute{\*M})$
% \ENDIF
% \STATE $\|\tilde{\grave{\*u}}_{i}\|_{2} = (\grave{\*a}_{i}^{T}\grave{\*M}_{I}\grave{\*a}_{i})^{1/2};\|\tilde{\acute{\*u}}_{i}\|_{2} = (\acute{\*a}_{i}^{T}\acute{\*M}_{I}\acute{\*a}_{i})^{1/2}$
% \STATE ${l}_{i} = \|\tilde{\grave{\*u}}_{i}\|_{2}\|\tilde{\acute{\*u}}_{i}\|_{2}$
% \STATE $p_{i} = \min\{cl_{i},1\}$
% \STATE Sample $\*a_{i}/\sqrt[p]{p_{i}}$ in $\*B$ with probability $p_{i}$
% \ENDWHILE
% \ELSE
% \STATE $\grave{\*M} = 0^{d^{p/2} \times d^{p/2}}; \grave{Q}$ column basis of $\grave{\*M}$
% \WHILE{$i \leq n$}
% \STATE $\grave{\*a}_{i} = \mbox{vec}(\*a_{i}\otimes^{p/2})$
% \STATE $[\tilde{l}_{i}, \grave{\*M}, \grave{\*Q}]$ = OnlineSesitvity($\grave{\*a}_{i},\grave{\*M},\grave{\*Q},p,r$)
% \STATE $L = L+\tilde{l}_{i}; p_{i} = \min\{r\tilde{l}_{i}/L,1\}$
% \STATE Sample $\*a_{i}/\sqrt[p]{p_{i}}$ in $\*C$ with probability $p_{i}$
\ENDWHILE
% \ENDIF
\end{algorithmic}
\end{algorithm}

As we mentioned earlier in, for even value of $p$ we have $\grave{\*a}_{i}=\acute{\*a}_{i}=\hat{\*a}_{i}$. Similarly the counterparts of $\*M$ and $\*Q$ are also same. We summarize the guarantees of \kernelfilter~in the following theorem.
\begin{theorem}{\label{thm:slowOnline}}
 Given a matrix $\*A \in \~R^{n \times d}$ whose rows are coming one at a time, the \kernelfilter~selects a set $\*C$ of size $O(\frac{d^{\lceil p/2 \rceil}k}{\epsilon^{2}}(1+\lceil p/2 \rceil\log \|\*A\|) - \frac{\lfloor p/2 \rfloor k}{\epsilon^2}\min_{i}\log \|\*a_{i}\|)$ with working space and update time $O(d^{p+1})$. Suppose $\*Q$ is a fixed k-dimensional subspace, then with probability atleast 0.9, with $\epsilon > 0, \forall \*x \in \*Q$ we have $\*C$ satisfying both tensor contraction and $\ell_{p}$ subspace embedding as in equation \eqref{eq:contract} and \eqref{eq:lp} respectively.
%  $$|\sum_{\hat{\*a}_{j}\in \*C} (\hat{\*a}_{j}^{T}\*x)^{p} - \sum_{\*a_{i}\in \*A} (\*a_{i}^{T}\*x)^{p}| \leq \epsilon \sum_{\*a_{i}\in \*A} |\*a_{i}^{T}\*x|^{p}$$
\end{theorem}
The working space and the computation time of the above algorithm are functions of $d$ and $p$. But compared to \online, \kernelfilter~ returns an asymptotically smaller coreset, which is independent of $n$. This is because the value $\tilde{l}_{i}$ gives a tighter upper bound of the online sensitivity score compared to what \online~gives.
% 
\subsection{\online+\kernelfilter}
Here we briefly sketch our final algorithm which achieves both the best sampling complexity as well as update time. We use our \online~along with \kernelfilter~to give a streaming algorithm that benefits both in space and time. For every incoming row \online~quickly decides its sampling probability and samples according to it and feeds it to \kernelfilter. We summarize the guarantee of \online+\kernelfilter,
\begin{theorem}{\label{thm:improvedOnlineCoreset}}
 Consider $\*A \in \~R^{n \times d}$ whose rows are coming one at a time. \online+\kernelfilter~takes $O(d^{2})$ amortized update time and uses $O(d^{p+1})$ working space to return $\*C$ of size $O((\frac{d^{\lceil p/2 \rceil}k}{\epsilon^{2}})(1+\lceil p/2 \rceil\log \|\*A\|) - \frac{\lfloor p/2 \rfloor k}{\epsilon^2}\min_{i}\log \|\*a_{i}\|))$ such that with at least 0.9 probability, $\epsilon > 0, \forall \*x \in \*Q$, $\*C$ satisfies both tensor contraction and $\ell_{p}$ subspace embedding as equation \eqref{eq:contract} and \eqref{eq:lp} respectively.
% $$|\sum_{\hat{\*a}_{i} \in \*C}(\hat{\*a}_{i}^{T}\*x)^{p} - \sum_{i\leq n}(\*a_{i}^{T}\*x)^{p}| \leq \epsilon\sum_{i\leq n}|\*a_{i}^{T}\*x|^{p}$$
\end{theorem}
Since the \online~only chooses a sublinear number of samples, which we feed to \kernelfilter, hence the amortized update time of \online+\kernelfilter~is same as the update time of \online.
% 
\subsection{$p=2$ case}
In case of matrix i.e. $p=2$ the \online~and \kernelfilter~are just the same. This is because for every incoming row $\*a_{i}$ the kernelization returns the same row as it is. Hence \kernelfilter's sampling process is exactly same as \online. While we use the sensitivity framework, for $p=2$ our proofs are novel in the following sense:
\begin{enumerate}
 \item  When creating the sensitivity scores in the online setting, we also do not need
 to use a regularization term as~\cite{cohen2016online}, instead relying on a novel analysis when the matrix is rank deficient. Hence
 for even-order tensors we get a relative error bound without making the number of samples depend on the minimum singular value (which~\cite{cohen2016online} need for online row sampling for matrices).
 \item We do not need to use a martingale based argument, since the sampling probability of a row does not depend on the previous samples. 
\end{enumerate}
Our algorithm gives a coreset which preserves relative error approximation (i.e. subspace embedding). Note that lemma 3.5 of \cite{cohen2016online} can be used to achieve the same but it requires the knowledge of $\sigma_{\min}(\*A)$(smallest singular value of $\*A$). There we need $\delta = \epsilon\sigma_{min}(\*A)$ which gives sampling complexity as $O(d\log(d)\log(\kappa(\*A))/\epsilon^{2})$. Our algorithm gives relative error approximation even when $\kappa(\*A) =1$, which is not clear in \cite{cohen2016online}. We state our guarantees in the appendix~\ref{app:matrix}.
% 
% In the following theorem we state the guarantees of our algorithm. 
% \begin{corollary}
% \label{lem:matrixcoreset}
%  Given a matrix $\*A \in \~R^{n\times d}$ with rows coming one at a time, for $p=2$ the algorithm \ref{alg:onlineCoreset} takes $O(d^{2})$ update time and samples $O(\frac{d}{\epsilon^{2}}(d+d\log\|\*A\|-\min_{i}\log \|\*a_{i}\|))$ rows and preserves the following with probability at least 0.9, $\forall \*x \in \~R^{d}$
%  $(1-\epsilon)\|\*A\*x\|^{2} \leq \|\*C\*x\|^{2} \leq (1+\epsilon)\|\*A\*x\|^{2}$.
% \end{corollary}
% % 
% Note that lemma 3.5 of \cite{cohen2016online} can be used to achieve the above guaranteed relative error approximation with the knowledge of $\sigma_{\min}(\*A)$(smallest singular value of $\*A$), but $\forall \*x \in \~R^{d}$. There we need $\delta = \epsilon\sigma_{min}(\*A)$ which gives sampling complexity as $O(d\log(d)\log(\kappa(\*A))/\epsilon^{2})$ to get the above guarantee. 
% 
% Just by using Matrix Bernstein inequality~\cite{tropp2011freedman} we can slightly improve the sampling complexity from $O(d^{2})$ to $O(d\log d)$. For simplicity we modify the sampling probability to $p_{i} = \min\{r\tilde{l}_{i},1\}$ and get the following guarantee.
% \begin{theorem}{\label{thm:improvedMatrixCoreset}}
%  The above modified algorithm samples $O(\frac{\log d}{\epsilon^{2}}(d+d\log\|\*A\|-\min_{i} \log \|\*a_{i}\|))$ rows and preserves the following with probability at least 0.9, $\forall \*x \in \~R^{d}$
%  \begin{align*}
%   (1-\epsilon)\|\*A\*x\|^{2} \leq \|\*C\*x\|^{2} \leq (1+\epsilon)\|\*A\*x\|^{2}
%  \end{align*}
% \end{theorem}
% Note that our algorithm gives relative error approximation even when $\kappa(\*A) =1$, which is not clear in \cite{cohen2016online}. We discuss the proof in the appendix ~\ref{proof:improvedMatrixCoreset}.
% 
% \subsection{Lower bound}
% In this section we show that there is a input sequence of rows $\*a$ such that the lemma \ref{lemma:onlineSummationBound} is tight in terms of $n^{1-2/p}$, i.e. there is an input such that $\sum_i \tilde{l}_i = \Omega(dn^{1 - 2/p})$.
% 
% Consider $d$ independent vectors $\*v_{1}, _{\cdots}, \*v_{d}$ and every streaming row $\*a_{i}$ is one of them. Let we get $n/d$ copies of each independent rows. For simplicity of the analysis we consider the same definition of $\tilde{l}_{i} = \min\{1,n^{p/2-1}c_{i}^{p/2}\}$ as in lemma \ref{lemma:onlineSummationBound}, where $1/c_{i}  = [n/d], \forall i \in [n]$. Now for the first $n/d$ rows we we have $c_{i} = 1/i$, so for $i \leq (n/d)^{1-2/p}$ we get $\tilde{l}_{i} = 1$. Therefore number of $\tilde{l}_{i} = 1$ is $\theta((n/d)^{1-2/p})$. Finally taking a union over all the $d$ independent vectors we get the number of $\tilde{l}_{i} = 1$ is $\theta(n^{1-2/p}d^{2/p})$.
\section{Proofs}{\label{sec:proofs}}
Here we state the supporting lemmas to prove our main theorems. We give a sketches of proof for some lemmas. The proofs are discussed in detail in the appendix~\ref{sec:appendix}.
\subsection{\online}
Here we give sketch of the proof for theorem \ref{thm:Online}. For ease of notation, the rows are considered numbered according to their order of arrival. The supporting lemmas are for the online setting which also work for the streaming case.
% When the $i^{th}$ row arrives, the algorithm has to decide whether to select it or not. 
% Clearly, at any intermediate step, we cannot compute the offline sensitivity scores. 
We show that it is possible to generalize the notion of sensitivity for the online setting as well as give an upper bound to it. We define the {\em online sensitivity} of any $i^{th}$ row as : $\sup_{\*x\in \*Q}\frac{|\*a_{i}^T\*x|^{p}}{\sum_{j=1}^{i}|\mathbf{a_{j}}^T\*x|^{p}} = 
\sup_{\*y\in \*Q'}\frac{|\*u_{i}^T\*y|^{p}}{\sum_{j=1}^{i}|\*u_{j}^T\*y|^{p}}$,
where $\*Q'= \{\*y| \*y = \Sigma \*V^T \*x, \*x\in \*Q\}$ and svd$(\*A) = \*U\Sigma \*V^{T}$. Notice that the denominator now contains a sum only over the rows that have arrived. 
We note that while online sampling results often need the use of martingales as an analysis tool, e.g.~\cite{cohen2016online}, in our setting, the sampling probability of each row does depend on the previous rows, but not on whether they were sampled or not. The sampling decision
of each row is independent. Hence, the application of Bernstein theorem~\ref{thm:bernstein} suffices.

We first show that the sampling of the incoming rows using $p_{i}$'s defined in \online, are upper bounds to the online sensitivity scores. 
% 
\begin{lemma}{\label{lemma:onlineSensitivityBound}}
 Consider $\*A \in \~R^{n \times d}$, whose rows are provided in an online manner to \online. Let $\tilde{l}_i = \min\{i^{p/2-1}(\*a_{i}^T\*M^{\dagger}\*a_{i})^{p/2},1\}$, and $\*M$ is a $d \times d$ matrix maintained by the algorithm. Then $\forall i \in [n]$,  $\tilde{l}_i$ satisfies $\tilde{l}_i \ge \sup_{\*x}\frac{|\*a_{i}^T\*x|^{p}}{\sum_{j=1}^{i}|\mathbf{a_{j}}^T\*x|^{p}}$.     
  %Here $\*u_{i}$ is the $i^{th}$ row of the orthonormal column space basis $\U$ of $A$.
\end{lemma}
The proof of this lemma is discussed in the appendix~\ref{proof:onlineSensitivityBound}. Although the $\tilde{l}_{i}$'s are computed very quickly but the algorithm gives a loose upper bound due to an extra charge of factor $i^{p/2-1}$. As $i$ increases the charge in each $\tilde{l}_{i}$ also increases. Now using these upper bounds help us prove the following result.
\begin{lemma}{\label{lemma:onlineGuarantee}}
Let $r$ provided to \online~be 
$(2k\sum_{j=1}^{n}\tilde{l}_{j})/\epsilon^{2}$. Let \online~return a coreset $\*C$. Then with probability at least 0.9, $\forall \*x \in \*Q$, 
$\*C$ satisfies the tensor contraction as in equation \eqref{eq:contract} and $\ell_{p}$ subspace embedding as in equation \eqref{eq:lp}.
% \begin{align*}
%     \big| \sum_{\mathbf{\hat{a}_i}\in \*C} (\mathbf{\hat{a}_i}^T \*x)^p - \sum_{i\le n} (\*a_{i}^T \*x)^p \big| \le  \epsilon \sum_{i\le n} |\*a_{i}^T \*x|^p.
% \end{align*}
\end{lemma}
The proof of this lemma is given in the appendix~\ref{proof:onlineGuarantee}. 
%The above result is for fixed $y$. It is not difficult to see that applying an $\epsilon-$net argument similar to the one as in the online setting we can achieve the guarantees of (eqn \ref{eqn:offlineGuarantee}), with $O\big(\frac{k\sum_{j=1}^{n}\tilde{l}_{j}} {\epsilon^{2}} \log(2/\epsilon)\log(1/\delta)\big)$ sampled rows.
% 
%Define the random variables $W_{i} = \{\frac{1}{p_{i}}(\*u_{i}^Ty)^{p} \mbox{ w.p. }p_{i}, 0 \mbox{ w.p. }(1-p_{i})\}$ with $p_{i}$'s as the ones defined in algorithm \ref{alg:onlineCoreset}. We have $E[W_i] = (\*u_{i}^Ty)^{p}$. We show that $\sum_i W_i$ is concentrated. 
%\begin{center}
%    $\mbox{Pr}\big(|\sum_i W_i - \sum_{j=1}^{n} (\mathbf{u_{j}}^Ty)^{p}| \geq \epsilon \sum_{j=1}^{n} |\mathbf{u_{j}}^T\*y|^{p}\big) \leq \exp\Big(\frac{-r\epsilon^{2}}{(1+\epsilon)\sum_{j=1}^{n}\tilde{l}_{j}}\Big)$
%\end{center}
% 
Now in order to bound the number of samples, we need a bound on the quantity $\sum_{j=1}^{n}\tilde{l}_{j}$ which we demonstrate in the following lemma.
\begin{lemma}{\label{lemma:onlineSummationBound}}
 %Given the rows of matrix $\*A \in \~R^{n \times d}$, coming in an online manner
 The $\tilde{l}_{i}$ in \online~algorithm which satisfies lemma \ref{lemma:onlineSensitivityBound} and lemma \ref{lemma:onlineGuarantee} has
 $\sum_{i=1}^{n}\tilde{l}_{i}=O(n^{1-2/p}(d+d\log \|\*A\| - \min_{i}\log \|\*a_{i}\|)$.
\end{lemma}
\begin{proof}{\label{sketch:onlineSummationBound}}
Here we give a sketch of the proof. The details are discussed in appendix~\ref{proof:onlineSummationBound}. First we bound $\sum_{i<\leq n} \tilde{e}_{i}$, where $\tilde{e}_{i}$ are the scores returned by $OnlineScore()$ (Function~\ref{alg:onineScore}) .
At time $i$ the algorithm gets $\*a_{i}$ and $\*M = \*A_{i-1}^T\*A_{i-1}$ using which it computes $\tilde{e}_i = \*a_{i}^T(\*A_{i}^T\*A_{i})^{\dagger}\*a_{i}$. 
With incoming row $\*a_{i}$ the rank of $\*M$ increases 
from $1$ to at most $d$. We say that the algorithm is
in phase-$k$ if the rank of $\*M$ equals $k$. For each phase $k \in [1, d-1]$, let $i_k$ denote the index
where row $\*a_{i_k}$ caused a phase-change in $\*M$ i.e. 
rank of $(\*A_{i_k-1}^{T} \*A_{i_k-1})$ is $k-1$, but rank of $(\*A_{i_k}^{T} \*A_{i_k})$ is $k$. 
Note that for each $i_k$, $\tilde{e}_{i_k} = 1$. There are at most $d$ such indices $i_k$.
% 
% We now bound the $\sum_{i\in [i_k, i_{k+1}-1]} \tilde{e}_i$. 
We bound the term $\tilde{e}_{i}, i \in [i_k+1,i_{k+1}-1]$. Suppose the $\mbox{thin-SVD}(\*A_{i_k}^T \*A_{i_k})=\*V\Sigma_{i_k} \*V^T$, all entries in $\Sigma_{i_k}$ being positive. We define $\*X_{i_k} = \*V^T (\*A_{i_k}^T \*A_{i_k}) \*V$, where $\*X_{i_k}$ is positive definite and hence full rank. Now for each $i\in [i_{k}+1, i_{k+1}-1]$ we have $\*a_{i} = \*V \*b_{i}$. We also have $\*X_{i} = \*X_{i-1} + \*b_{i} \*b_{i}^T$. So we have, $\tilde{e}_{i} = \*a_{i}^T(\*A_{i}^T \*A_{i} )^{\dagger}\*a_{i} = \*b_{i}^T\*V^T(\*V(\mathbf{\Sigma_{i-1}}+\*b_{i} \*b_{i}^T)\*V^T)^{\dagger}\*V \*b_{i}=\*b_{i}^T(\*X_{i-1}+\*b_{i}\*b_{i}^T)^{\dagger}\*b_{i} = \*b_{i}^T(\*X_{i-1}+\*b_{i}\*b_{i}^T)^{-1}\*b_{i}$.
%  
% % Note that matrix $\mathbf{\Sigma_{i-1}}+\*b_{i}\*b_{i}^T$ is a positive definite and hence a full rank matrix. 
 Now using the matrix determinant lemma~\cite{vrabel2016note} on % the above term we get,
 $\mbox{det}(\*X_{i-1}+\*b_{i}\*b_{i}^T)$ we get,
 \begin{eqnarray*}
  &=& \mbox{det}(\*X_{i-1})(1+\*b_{i}^T(\*X_{i-1})^{-1}\*b_{i})  \\
  &\geq& \mbox{det}(\*X_{i-1})(1+\*b_{i}^T(\*X_{i-1}+\*b_{i} \*b_{i}^T)^{-1}\*b_{i})  \\
  &=& \mbox{det}(\*X_{i-1})(1+\tilde{e}_{i}) \geq \mbox{det}(\*X_{i-1})\exp(\tilde{e}_{i}/2)
%   \exp(\tilde{e}_{i}/2) &\leq& \frac{\mbox{det}(\*X_{i-1}+\*b_{i}\*b_{i}^T)}{\mbox{det}(\*X_{i-1})} 
 \end{eqnarray*}
 So finally we get $\exp(\tilde{e}_{i}/2) \leq \frac{\mbox{det}(\*X_{i-1}+\*b_{i}\*b_{i}^T)}{\mbox{det}(\*X_{i-1})}$. Now computing other $\tilde{e}_{i}$'s and by a careful analysis we get a telescopic sum, which simplifies $\sum_{i\leq n}\tilde{e}_{i} \leq \tilde{e}_{1} - \tilde{e}_{n}$.
\end{proof}
With lemmas~\ref{lemma:onlineSensitivityBound}, \ref{lemma:onlineGuarantee} and \ref{lemma:onlineSummationBound} we prove that the guarantee in theorem \ref{thm:Online} is achieved by \online. The bound on space is evident from the fact that we are maintaining the matrix $\*M$ in algorithm which contributes $d^{2}$ and returns a sample of size $O(\frac{n^{1-2/p}k}{\epsilon^{-2}}(d+d\log\|\*A\|-\min_{i}\log \|\*a_{i}\|))$. 
% 
\subsection{\kernelfilter}
In this section we give sketch of the proof of theorem \ref{thm:slowOnline}. Here also we use sensitivity based framework to decide sampling probability of each incoming row. The novelty in this algorithm is by reducing the $p$ order tensor problem to a $2$ order tensor. Now we give bound on sensitivity score of every incoming row.
% 
\begin{lemma}{\label{lemma:slowOnlineSensitivityBound}}
 Consider $\*A \in \~R^{n \times d}$ rows are coming in online manner to \kernelfilter. The term $\tilde{l}_{i}$ defined in the algorithm upper bounds the online sensitivity score, i.e. $\sup_{\*x}\frac{|\*a_{i}^{T}\*x|^{p}}{\sum_{j=1}^{i}|\*a_{j}^{T}\*x|^{p}}$
\end{lemma}
% 
The proof is discussed in the appendix~\ref{proof:slowOnlineSensitivityBound}. It gives tighter upper bounds to the sensitivity scores compared to lemma~\ref{lemma:onlineSensitivityBound}. The tightness will be evident when we sum these upper bounds. Next we show with these $\tilde{l}_{i}$'s we can claim the following,
\begin{lemma}{\label{lemma:slowOnlineGuarantee}}
 In the \kernelfilter~let $r$ is set as $O(k\sum_{i=1}^{n}\tilde{l}_{i}/\epsilon^{2})$ then for some fixed k-dimensional subspace $\*Q$ the set $\*C$ with probability 0.9 $\forall \*x \in \*Q$ satisfies tensor contraction as in equation \eqref{eq:contract} and $\ell_{p}$ subspace embedding as in equation \eqref{eq:lp}.
%  $$|\sum_{\hat{\*a}_{i}\in\*C}(\hat{\*a}_{i}^{T}\*x)^{p} - \sum_{i\leq n} (\*a_{i}^{T}\*x)^{p}| \leq \sum_{i\leq n} |\*a_{i}^{T}\*x|^{p}$$
\end{lemma}
% 
We discuss this in detail in the appendix~\ref{proof:slowOnlineGuarantee}. Next we bound the sum of upper bounds, i.e. $\sum_{i \leq n} \tilde{l}_{i}$.
\begin{lemma}{\label{lemma:slowOnlineSummationBound}}
 The $\tilde{l}_{i}$'s used in \kernelfilter~which satisfy lemma \ref{lemma:slowOnlineGuarantee} has
 $\sum_{i=1}^{n} \tilde{l}_{i}$ is $O(d^{\lceil p/2 \rceil}(1+\lceil p/2 \rceil\log \|\*A\|) - \lfloor p/2 \rfloor\min_{i}\log \|\*a_{i}\|)$.
\end{lemma}
% 
We discuss the proof in detail in the appendix~\ref{proof:slowOnlineSummationBound}. The proof of the above lemma is similar to that of lemma \ref{lemma:onlineSummationBound}. It implies that the lemma \ref{lemma:slowOnlineSensitivityBound} gives tighter sensitivity bounds compared to lemma \ref{lemma:onlineSensitivityBound}. 

With lemmas \ref{lemma:slowOnlineSensitivityBound}, \ref{lemma:slowOnlineGuarantee} and \ref{lemma:slowOnlineSummationBound} we prove that the guarantee in theorem \ref{thm:slowOnline} is achieved by algorithm \ref{alg:slowOnline}. The bound on space required is evident from the fact that we are maintaining a $d^{p/2} \times d^{p/2}$ matrix for even $p$ hence working space is $O(d^p)$ and for odd value of $p$ it is $O(d^{p+1})$ because the algorithm maintains a $d^{\lceil p/2 \rceil} \times d^{\lceil p/2 \rceil}$ matrix. 
\input{sections/application.tex}
% \subsection{Single Topic Model}\label{exp}
Here we empirically show how sampling using \online+\kernelfilter~can preserve tensor contraction as in equation \eqref{eq:contract}. This can be used in single topic modeling where documents are coming in streaming manner. We compare our method with 2 other sampling schemes, namely -- uniform and online leverage scores which we call \online$(2)$.

% \textbf{Real dataset:} We used a subset of 20 Newsgroups data(pre-processed). We took a subset of 8k documents and considered the 50 most frequent words. We normalized each document vector to $\ell_{1}=1$ and ran Algorithm \ref{alg:onlineCoreset} with $p=3$ on it. Next we created $\mathcal{T}$ and $\hat{\mathcal{T}}$ as defined above and we report $|\mathcal{T}(\*x,\*x,\*x) - \hat{\mathcal{T}}(\*x,\*x,\*x)|/(|\mathcal{T}(\*x,\*x,\*x)|)$ values of the different algorithms. In order to capture the query $\mathbf{x}$ which will be ``most affected'' (i.e. have high variance) due to sampling, we chose $\mathbf{x}$ as the right singular vector corresponding to the smallest singular value of the above mentioned dataset. Here the sample size is over expectation.
\textbf{Real dataset:} We used a subset of 20Newsgroups data(pre-processed). We took a subset of 10k documents and considered the 100 most frequent words. We normalized each document vector to $\ell_{1}$ norm $1$ and created a matrix $\*A \in \~R^{10k \times 100}$. We feed its row at a time to \online+\kernelfilter with $p=3$, which returns a coreset $\*C$. For a dataset with 12 topics, we run tensor based single topic modeling~\cite{anandkumar2014tensor} on $\*A$ and $\*C$, which returns 12 topic distributions for both. Next we take the best matching between empirical and estimated topics based on $\ell_{1}$ distance and  compute the average $\ell_{1}$ difference between them. We run this entire method 5 times and report the median of the $\ell_{1}$ differences. Here the coreset sizes are over expectation.

\begin{table}[htbp]
 \caption{20NewsGroup-Single Topic Modeling}
 \label{tab:empCompare}
 \vskip 0.1in
 \begin{center}
  \begin{scriptsize}
   \begin{sc}
    \begin{tabular}{|c|c|c|c|}
     \hline
     Sample & \uni & \online$(2)$ & \online\\
     & & & +\kernelfilter \\
     \hline
     50 & 0.5725 & 0.6903 & \textbf{0.5299}  \\
     \hline
     100 & 0.5093 & 0.6385 & \textbf{0.4379} \\
     \hline
     200 & 0.4687 & 0.5548 & \textbf{0.3231} \\
     \hline
     500 & 0.3777 & 0.3992 & \textbf{0.2173} \\
     \hline
     1000 & 0.2548 & 0.2318 & \textbf{0.1292} \\
     \hline
    \end{tabular} % &
   \end{sc}
  \end{scriptsize}
 \end{center}
 \vskip -0.1in
\end{table}
% The expected size of the coreset is controlled by parameter $r$ of the algorithm. Next we created $\mcal T = \sum_{i=1}^{n} \*W\*a \otimes^{3}$ and $\widetilde{\mcal T} = \sum_{\tilde{\*a}_{i} \in \*C} \*W\tilde{\*a}_{i}\otimes^{3}$. as defined above and we report $|\mathcal{T}(\*x,\*x,\*x) - \hat{\mathcal{T}}(\*x,\*x,\*x)|/(|\mathcal{T}(\*x,\*x,\*x)|)$ values of the different algorithms. In order to capture the query $\mathbf{x}$ which will be ``most affected'' (i.e. have high variance) due to sampling, we chose $\mathbf{x}$ as the right singular vector corresponding to the smallest singular value of the above mentioned dataset. Here the sample size is over expectation.
% \begin{table}[htbp]
%     \centering
%     \begin{tabular}{|c|c|c|c|}
%         \hline
%         Sample & Uniform & Leverage & Sensitivity \\
%         \hline
%         200 & 3.306 & 1.357 & \textbf{1.003}  \\
%         \hline
%         350 & 2.045 & 0.925 & \textbf{0.921} \\
%         \hline
%         450 & 1.212 & 1.119 & \textbf{0.735} \\
%         \hline
%         600 & 2.852 & 1.223 & \textbf{0.647} \\
%         \hline
%     \end{tabular} % &
%     \caption{Tensor contraction on 20 Newsgroups dataset. \\(The numbers reported are the average of 10 experiments for each sample size.)}
%     \label{tab:realData}
% \end{table}
It can be seen from the tables, that our algorithm \online+\kernelfilter~performs better or at par with both uniform and \online$(2)$ thus supporting our theoretical claims. 
% There are no theoretical claims for leverage score sampling for tensor factorization.
% \section{Conclusion and Discussion}
We give a one pass algorithm that samples rows coming in an online manner and maintains a coreset which preserves properties essential for tensor factorization. The sampling complexity and space required for the algorithm is sublinear in $n$. We describe the application of our algorithm to the case of single topic modeling. We empirically show the effectiveness of our algorithm as compared to other sampling techniques. The obvious extensions of this work are i) to give tighter bounds for the sum of sensitivities and hence decrease the dependence of the sampling complexity on $n$ and ii) decrease the space usage. Our work is focused on supersymmetric tensors. Another obvious open question is to modify this algorithm for general tensors.


% In the unusual situation where you want a paper to appear in the
% references without citing it in the main text, use \nocite
% \nocite{langley00}

\bibliography{refer}
\bibliographystyle{icml2020}


%%%%%%%%%%%%%%%%%%%%%%%%%%%%%%%%%%%%%%%%%%%%%%%%%%%%%%%%%%%%%%%%%%%%%%%%%%%%%%%
%%%%%%%%%%%%%%%%%%%%%%%%%%%%%%%%%%%%%%%%%%%%%%%%%%%%%%%%%%%%%%%%%%%%%%%%%%%%%%%
% DELETE THIS PART. DO NOT PLACE CONTENT AFTER THE REFERENCES!
%%%%%%%%%%%%%%%%%%%%%%%%%%%%%%%%%%%%%%%%%%%%%%%%%%%%%%%%%%%%%%%%%%%%%%%%%%%%%%%
%%%%%%%%%%%%%%%%%%%%%%%%%%%%%%%%%%%%%%%%%%%%%%%%%%%%%%%%%%%%%%%%%%%%%%%%%%%%%%%
\clearpage

\appendix
\section{Appendix}{\label{sec:appendix}}
Before we discuss the proofs of the supporting lemmas and arguments, we state definition and theorems that are useful in the discussion.
\paragraph{Well-Conditioned Basis:\label{defi:wellConditionedBais}}\cite{dasgupta2009sampling}Let $\*A \in \~R^{n \times d}$ a rank $d$ matrix. For $p \geq 1$, it has a dual $q = p/(p-1)$. A matrix $\*U$ is said to be $(\alpha,\beta,p)$ \emph{well-conditioned basis} of $\*A$, if $\*U$ spans the column space of $\*A$, $\sum_{i=1}^{d}\|\*u_{j}\|_{p}\leq \alpha$, $\forall \*x \in \~R^{d}, \frac{\|\*x\|_{q}}{\|\*U\*x\|_{p}}\leq \beta$ and $(\alpha, \beta)$ are $d^{O(1)}$ and also independent of $n$.
%Given a matrix $A \in \~R^{n \times d}$, 
%we represent $u_{i}$ as the $i^{th}$ row vector of the orthonormal basis of $\*A$. 
%In the online setup, $A_{i}$ will denote a a $\~R^{i \times d}$ matrix consisting of first $i$ rows seen so far. %At any $i^{th}$ step $\tilde{\*u}_{i}$ represents the $i^{th}$(i.e. last) row vector of the orthonormal column basis of $A_{i}$. Note that the $\tilde{\*u}_{i}$ corresponds to $A_{i}$ only. To clarify, $\tilde{u}_{i-1}$ is the $(i-1)^{th}$ row of orthonormal basis of $A_{i-1}$ and is not the same as the  $(i-1)^{th}$ row of orthonormal basis of $A_{i}$.
% 
%$p$-order \emph{rank-1} tensor is defined as an outer product of $p$ vectors, 
%i.e. $\T\in  \~R^{n_{1} \times n_{2} \times n_{3}}$ is rank-$1$ if there exist
%vectors $a \in \~R^{n_{1}}, b \in \~R^{n_{2}}$ and $c \in \~R^{n_{3}}$ such that 
%$\T = a \otimes b \otimes c$. We will only deal with \emph{supersymmetric} tensors, which 
%are defined as $\T = a\otimes a\otimes a$ for some $a\in \Re^n$.  Note that $\T = a\otimes 3$, where $a \in \~R^{n}$ is a super symmetric rank-1 tensor. A rank-$r$ tensor is a sum of $r$ rank-1 tensors.
% 
%For a tensor $\T \in \~R^{n \times n \times n}$, the \emph{tensor contraction} operations that is mostly used in this work is $\T(x,x,x) = \sum_{i=1}^{n}(a_{i}^{T}x)^{3}$, which results to a scalar.
% 
% Based on tensor contraction operation we define tensor subspace embedding for a super-symmetric tensor. Given a tensor $\mcal T \in \~R^{d\times d \times d}$ the $(1 \pm \epsilon)$ subspace embedding is $\widehat{\mcal T} \in \~R^{d\times d \times d}$, which $\forall \*x \in \~R^{d}$, $\widehat{\mcal T}(x,x,x)=(1\pm\epsilon)\mcal T(x,x,x)$.
% 
\begin{theorem}{\label{thm:bernstein}}
\textbf{Bernstein \cite{dubhashi2009concentration}} Let the scalar random variables $x_{1}, x_{2}, _{\cdots}, x_{n}$ be independent that satisfy $\forall i \in [n]$,  
$\vert x_{i}-\~E[x_{i}]\vert \leq b$. 
Let $X = \sum_{i} x_{i}$ and let $\sigma^{2} = \sum_{i} \sigma_{i}^{2}$ be the variance of $X$. 
Then for any $t>0$,
\begin{center}
 $\mbox{Pr}\big(X > \~E[X] + t\big) \leq \exp\bigg(\frac{-t^{2}}{2\sigma^{2}+bt/3} \bigg)$
\end{center}
\end{theorem}
% 
\begin{theorem}{\label{thm:matrixBernstein}}
 \textbf{Matrix Bernstein\cite{tropp2015introduction}} Let $\*X_{1},\ldots,\*X_{n}$ are independent $d \times d$ random matrices such that $\forall i \in [n], \|\*X_{i}\|] \leq b$ and $\mbox{var}(\|\*X\|) \leq \sigma^{2}$ where $\*X = \sum_{i=1}^{n}\*X_{i}$, then for some $t>0$,
 $$\mbox{Pr}(\|\*X\| - \~E[\|\*X\|] \geq t) \leq d\exp(\frac{-t^{2}/2}{\sigma^{2}+bt/3})$$
\end{theorem}
% 
\begin{lemma}{\label{lemma:modified-SM}}
 Given a rank-k positive semi-definite matrix $\*M \in \~R^{d \times d}$ and a vector $\*x$ such that it completely lies in the column space of $\*M$. Then we have,
 $$(\*M + \*x\*x^{T})^{\dagger} = \*M^{\dagger} - \frac{\*M^{\dagger}\*x\*x^{T}\*M^{\dagger}}{1+\*x^{T}\*M^{\dagger}\*x} $$
\end{lemma}
\begin{proof}
 The proof is in the similar spirit to lemma \ref{lemma:onlineSummationBound}. Consider $[\*V\Sigma,\*V] = \mbox{SVD}(\*M)$ and since $\*x$ lies completely in the column space of $\*M$, hence $\exists \*y \in \~R^{k}$ such that $\*V\*y = \*x$. Note that $\*V \in \~R^{d \times k}$.
 \begin{eqnarray*}
  (\*M + \*x\*x^{T})^{\dagger} &=& (\*V\Sigma\*V^{T}+\*V\*y\*y^{T}\*V^{T})^{\dagger} \\
  &=& \*V(\Sigma+\*y\*y^{T})^{-1}\*V^{T} \\
  &=& \*V\bigg(\Sigma^{-1} - \frac{\Sigma^{-1}\*y\*y^{T}\Sigma^{-1}}{1+y^{T}\Sigma^{-1}\*y}\bigg)\*V \\
  &=& \*V\bigg(\Sigma^{-1}-\frac{\Sigma^{-1}\*V^{T}\*V\*y\*y^{T}\*V^{T}\*V\Sigma^{-1}}{1+y^{T}\*V^{T}\*V\Sigma^{-1}\*V^{T}\*V\*y}\bigg)\*V\\
  &=& \*M^{\dagger} - \frac{\*M^{\dagger}\*x\*x^{T}\*M^{-1}}{1+\*x\*M^{\dagger}\*x}
 \end{eqnarray*}
 The third equality is by Sherman Morrison Formula.
\end{proof}
% 
% Now we discuss the proofs of the supporting lemmas and arguments. 
% \subsection{Offline Coreset}
% Here we present the offline method for completeness. The set $\*C$ returns a set of vectors $\hat{\*a}_{j}$ which are original vectors $\*a_{j}$ with appropriate scaling.
% \begin{algorithm}[htpb]
% \caption{Online Sampling for $p-$mode tensor}{\label{alg:onlineCoreset}}
% \begin{algorithmic}[1]
% \REQUIRE Matrix $\*A \in \~R^{n \times d}; p>0$, $r > 1$.
% \ENSURE Coreset $\*C$ satisfying eqn \ref{eq:contract} whp
% \STATE $\*U = \mbox{well-conditioned-basis}(\*A)$
% \WHILE{current row $\*a_{i}^T$ is not the last row}
% \STATE $p_{i} = \min\{\frac{r\|\*u_{j}^{T}\|_{p}^{p}}{\big|\|\*U\|\big|_{p}^{p}},1\}$
% \STATE Sample $\*a_{j}/\sqrt[p]{p_{i}}$ in $\*C$ with probability $p_{i}$
% \ENDWHILE
% \STATE Return $\*C$
% \end{algorithmic}
% \end{algorithm}
% % 
% \subsubsection{Proof of Lemma \ref{thm:Offline}}
% \begin{proof}{\label{proof:offline}}
% The proof is mostly similar to the proof of theorem 5 in \cite{dasgupta2009sampling}. There for the given matrix $\*A$ the authors consider $(\alpha,\beta,p)$ well conditioned basis as defined above, where $\alpha = d^{1/p+1/2}$ and $\beta = d^{1/q-1/2}$. Such that $\*A = \*U{\tau}$ and $\*a_{i}^{T}\*x = \*u_{i}^{T}\tau \*x$. We define a random variable 
%  \[ x_{i} =
%   \begin{cases}
%     \frac{1}{p_{i}}(\*a_{i}^{T}\*x)^{p}  & \quad \text{with probability } p_{i}\\
%     0 & \quad \text{with probability } (1-p_{i})
%   \end{cases}
% \]
% Here we have $\~E[x_{i}] = (\*a_{i}^{T}\*x)^{p}$. Now using Bernstein's theorem \ref{thm:bernstein} one can show that the sum $\sum_{i}^{n} x_{i}$ is close to $\sum_{i=1}^{n} \~E[x_{i}]$ with some constant probability. First we bound the term $|x_{i} - \~E[x_{i}]| \leq b, \forall i \in [n]$. Note that if $x_{i}$ is non zero then,
% \begin{eqnarray*}
%  |x_{i} - \~E[x_{i}]| &\leq& |x_{i}| = |(\*a_{i}^{T}\*x)^{p}|/p_{i} \\
%  &=& |\*a_{i}^{T}\*x|^{p}/p_{i} \leq \|\*u_{i}^{T}\|_{p}^{p}\cdot \|\tau \*x\|_{q}^{p}/p_{i} \\
%  &\leq& \|\*U\|_{p}^{p}\cdot \|\tau \*x\|_{q}^{p}/r \leq (\alpha\beta)^{p} \|\*A\*x\|_{p}^{p}/r
% \end{eqnarray*}
% Now when $x_{i}=0$, then $|x_{i} - \~E[x_{i}]| = |\~E[x_{i}]|$. Now this case can only happen when $p_{i} < 1$, i.e. 
% \begin{eqnarray*}
%  1 &>& \frac{r\|\*u_{i}^{T}\|_{p}^{p}}{|\|\*U\||_{p}^{p}} = \frac{r\|\*u_{i}^{T}\|_{p}^{p}\|\tau \*x\|_{q}^{p}}{|\|\*U\||_{p}^{p}\|\tau \*x\|_{q}^{p}} \\
%  &\geq& \frac{r\|\*u_{i}^{T}\|_{p}^{p}\|\tau \*x\|_{q}^{p}}{(\alpha\beta)^{p}\|\*A \*x\|_{p}^{p}} \geq \frac{r|\*a_{i}^{T}\*x|_{p}^{p}}{(\alpha\beta)^{p}\|\*A \*x\|_{p}^{p}} 
% \end{eqnarray*}
% So we get $|\*a_{i}^{T}\*x|^{p} \leq |\*a_{i}^{T}\*x|^{p}/p_{i} \leq (\alpha\beta)^{p}\|\*A \*x\|_{p}^{p}/r = b$.

% Next we bound the variance of $\sum_{i=1}^{n} x_{i}$, i.e $\sigma^{2} = \sum_{i=1}^{n}\mbox{var}(x_{i}) \leq \sum_{i;p_{i}<1}\~E[x_{i}^{2}]$,
% \begin{eqnarray*}
%  \sum_{i;p_{i}<1}\~E[x_{i}^{2}] &=& \sum_{i;p_{i}<1} |\*a_{i}^{T}\*x|^{2p}/p_{i} \\
%  &\leq& (\alpha\beta)^{p}\|\*A \*x\|_{p}^{p}/r \sum_{i;p_{i}<1} |\*a_{i}^{T}\*x|^{p} \\
%  &=& (\alpha\beta)^{p}\|\*A \*x\|_{p}^{2p}/r
% \end{eqnarray*}
% Note that $(\alpha\beta)^{p}$ is $d^{p}$. Now applying Bernstein's theorem \ref{thm:bernstein} on the following event. Let $t = \epsilon \sum_{i=1}^{n} |\*a_{i}^{T}\*x|^{p}$
% $$p = \mbox{Pr}\bigg(|\sum_{i=1}^{n}x_{i} - \sum_{i=1}^{n} (\*a_{i}^{T}\*x)^{p}| \geq \epsilon \sum_{i=1}^{n} |\*a_{i}^{T}\*x|^{p}| \bigg)$$
% \begin{eqnarray}
%  p &\leq& \exp\bigg(\frac{-\big(\epsilon \sum_{i=1}^{n} |\*a_{i}^{T}\*x|^{p}\big)^{2}}{\sigma^{2}+bt}\bigg) \nonumber \\ 
%   &\leq& \exp\bigg(\frac{-\big(\epsilon \sum_{i=1}^{n} |\*a_{i}^{T}\*x|^{p}\big)^{2}}{d^{p}\|\*A \*x\|_{p}^{2p}+\epsilon d^{p}\|\*A \*x\|_{p}^{2p}}\bigg) \nonumber\\
%   &=& \exp\Bigg(\frac{-r\epsilon^{2}}{(1+\epsilon)d^{p}}\Bigg) \nonumber
% \end{eqnarray}
% Now ensuring the above event occurs with at most $\delta$ probability and taking union bound over an $\epsilon$ net in $\~R^{d}$ we get the desired sampling complexity.
% 
% From theorem 6 in \cite{dasgupta2009sampling} we can create the well conditioned basis in $O(nd^{5}\log n)$. The running time of the algorithm \ref{alg:offlineCoreset} is dominated by the time required to create such well conditioned basis. 
% \end{proof}
% 
\noindent\textbf{$\epsilon$-net argument:}{\label{argument:epsNet}} Note that $\norm{\Pi \*A\*x}_{p}^{p} = \sum_{i=1}^{m} |\tilde{\*a}_{i}^{T}\*x|^{p}$, where $\Pi$ is a sampling matrix which samples $m$ rows from $\*A$ with proper scaling. Now we argue this $\forall \*x \in \*C$, i.e. $\norm{\Pi \*A\*x}_{p}^{p} = (1\pm\epsilon)\norm{\*A\*x}_{p}^{p}$ which is same as $\norm{\Pi \*U\*y}_{p}^{p} = (1\pm\epsilon)\norm{\*U\*y}_{p}^{p}$ where $\*U\*y=\*A\*x$. Now with an $\epsilon-$net, we argue $\forall \*x \in \*C$, $\*C \subseteq \~R^{d}$. 

Let $\mcal B = \{ \*z \in \~R^{n} \vert \*z = \*U\*y \mbox{ for some } y \in \~R^{k} \mbox{ and } \norm{\*z}_{p} = 1 \}$. From this set we intend to find a finite subset $\mcal N$ which is an $\epsilon$-net to the set. Now here we argue that if we can ensure $\norm{\Pi \*w}_{p}^{p} = (1\pm\epsilon)\norm{\*w}_{p}^{p}, \forall \*w \in \mcal N$ then we can claim that $\norm{\Pi \*z}_{p}^{p} = (1\pm\epsilon)\norm{\*z}_{p}^{p}, \forall \*z \in \mcal B$ which further imply that $\norm{\Pi \*A\*x}_{p}^{p} = (1\pm\epsilon)\norm{\*A\*x}_{p}^{p}, \forall \*x \in \*C$.

Let $\*v \in \mcal B$ whose closest $\epsilon$-net point is $\*w_{1} \in \mcal N$ such that $\norm{\*v - \*w_{1}}_{p} \leq \epsilon$. Now note that,
\begin{eqnarray*}
 \norm{\Pi \*v}_{p}^{p} &=& \norm{\Pi \*w_{1} + \Pi(\*v-\*w_{1})}_{p}^{p} \nonumber \\
 &\leq& (\norm{\Pi \*w_{1}}_{p} + \norm{\Pi(\*v-\*w_{1})}_{p})^{p} \nonumber \\
 &\leq& (1+\epsilon + \norm{\Pi(\*v-\*w_{1})}_{p})^{p} \nonumber \\
 &=& (1+\epsilon + \norm{\Pi(\*w_{2}/\alpha + \*v-\*w_{1}-\*w_{2}/\alpha)}_{p})^{p} \nonumber \\
 &\leq& (1+\epsilon + \epsilon(1+\epsilon) + \norm{\Pi(\*v-\*w_{1}-\*w_{2}/\alpha)}_{p})^{p}
\end{eqnarray*}
% 
Repeated application of this argument yields
% 
\begin{center}
 $\norm{\Pi \*v}_{p}^{p} \leq \bigg(\sum_{i\geq0} (1+\epsilon)\epsilon^{i}\bigg)^{p} \leq \bigg(\frac{1+\epsilon}{1-\epsilon}\bigg)^{p} \leq 1 + \mcal O(\epsilon)$ 
\end{center}
% 
By similar argument one can show that $\norm{\Pi \*v}_{p}^{p} \geq 1 - \mcal O(\epsilon)$. Finally by rescaling $\epsilon$ by some constant factor we achieve $\norm{\Pi \*z}_{p}^{p} \in 1\pm\epsilon, \forall \*z \in \mcal B$.
\begin{lemma}{\label{lemma:netSize}}
 There is an $\epsilon$-net $\mcal N$, with $\vert\mcal N\vert \leq (2/\epsilon)^{k}$.
\end{lemma}
\begin{proof}
Let $\mcal N$ be the maximal subset of $\*y \in \~R^{n}$ in the column space of $\*A$ such that $\norm{\*y}_{p} = 1$ and $\forall \*y \neq \*y' \in \mcal N, \norm{\*y-\*y'}_{p} > \epsilon$. Now as $\mcal N$ is a maximal set, hence $\forall \*y \in \mcal B, \exists \*w \in \mcal N$ for which $\norm{\*w-\*y}_{p} \leq \epsilon$. Further $\forall \*y \neq \*y' \in \mcal N$ two balls centered at $\*y$ and $\*y'$ with radius $\epsilon/2$ are disjoint otherwise by triangle inequality, $\norm{\*y-\*y'}_{p} \leq \epsilon$, is a contradiction. So it follows that in a unit sphere in $\~R^{k}$ there could be at most $(2/\epsilon)^{k}$ such balls, i.e. the number of points in $\mcal N$. 
\end{proof}
% 
% \subsubsection{Proof of Lemma \ref{lemma:offlineSummationBound}}
% \begin{proof}{\label{proof:summationBound}}
%  We know that $\sum_{i=1}^{n} \norm{u_{i}}^{2} = d$. Let the upper bound to an $S_{i}$ is $l_{i} = \min\{1,n^{p/2-1}\norm{u_{i}}^{p}\}$.  There are two cases that decide the value of $l_{i}$. When $\norm{u_{i}}^{p} \geq n^{1-p/2}$ then $l_{i} = 1$, this implies that $\norm{u_{i}}^{2} \geq n^{2/p-1}$. But we know $\sum_{i=1}^{n}\norm{u_{i}}^{2} = d$ so there are at-most $n^{1-2/p}d$ such $l_{i}$'s. Now for the case where $\norm{u_{i}}^{p} < n^{1-p/2}$, we get $\norm{u_{i}}^{p-2} < (1/n)^{(p/2-1)(1-2/p)}$. Then $\sum_{i=1}^{n} n^{p/2-1}\norm{u_{i}}^{p} = \sum_{i=1}^{n} n^{p/2-1}\norm{u_{i}}^{p-2} \norm{u_{i}}^{2} <\sum_{i=1}^{n}n^{1-2/p}\norm{u_{i}}^{2} = n^{1-2/p}d$.
% \end{proof}
% Note that (eqn \ref{eqn1}) for even $p$ value is same as subspace embedding. \cite{dasgupta2009sampling} shows that there is a sampling algorithm for any $\ell_{p}$ regression. To prove it authors first show that with $O(d^{p+1})$ one can preserve the $p$ norm subspace embedding. Now we discuss the lemmas of online setting.
\subsection{\online}
 Here we provide the proofs of the lemmas used to prove theorem~\ref{thm:Online}.
\subsubsection{Proof of lemma \ref{lemma:onlineSensitivityBound}}
\begin{proof}{\label{proof:onlineSensitivityBound}}
We define the online sensitivity scores $\tilde{s}_{i}$ for each point $i$ as follows,
\begin{center}
 $\tilde{s}_{i} = \sup_{\*x}\frac{\vert \*a_{i}^{T}\*x\vert^{p}}{\sum_{j=1}^{i}\vert \*a_{j}^{T}\*x\vert^{p}} = \sup_{\*y}\frac{\vert \*u_{i}^{T}\*y\vert^{p}} {\sum_{j=1}^{i}\vert \*u_{j}^{T}\*y\vert^{p}}$
\end{center}
Here $\*y = \Sigma \*V^{T}\*x$ where $[\*U,\Sigma,\*V] = \mbox{svd}(\*A)$ and $\*u_{i}^{T}$ is the $i^{th}$ row of $\*U$. Now at this $i^{th}$ step we define $[\*U_{i},\Sigma_{i},\*V_{i}] = \mbox{svd}(\*A_{i})$. So we rewrite the above optimization function as follows with $\*y = \Sigma_{i} \*V_{i}^{T}\*x$ and $\tilde{\*u}_{i}^{T}$ is the $i^{th}$ row of $\*U_{i}$,
\begin{center}
 $\tilde{s}_{i} = \sup_{\*x}\frac{\vert \*a_{i}^{T}\*x\vert^{p}}{\sum_{j=1}^{i}\vert \*a_{j}^{T}\*x\vert^{p}}=\sup_{\*y}\frac{\vert \tilde{\*u}_{i}^{T}\*y\vert^{p}}{\norm{\*U_{i}\*y}_{p}^{p}}=\sup_{\*y}\frac{\vert \tilde{\*u}_{i}^{T}\*y\vert^{p}}{\vert \tilde{\*u}_{i}^{T}\*y\vert^{p}+\sum_{j=1}^{i-1}\vert \tilde{\*u}_{j}^{T}\*y\vert^{p}}$
\end{center}
Let there be an $\*x^{*}$ which maximizes $\tilde{s}_{i}$. Corresponding to it we have  $\*y^{*} = \Sigma_{i} \*V_{i}^{T}\*x^{*}$. For a fixed $\*x$, let $f(\*x) = \frac{\vert \*a_{i}^{T}\*x\vert^{p}}{\sum_{j=1}^{i}\vert \*a_{j}^{T}\*x\vert^{p}}$ and $g(\*y) = \frac{\vert \tilde{\*u}_{i}^{T}\*y\vert^{p}}{\norm{\*U_{i}\*y}_{p}^{p}}$. By assumption we have $f(\*x^{*}) \geq f(\*x), \forall \*x$. We prove by contradiction that its corresponding $g(\*y^{*}) \geq g(\*y), \forall \*y$, where $\*y = \Sigma_{i} \*V_{i}^{T}\*x$. Let $\exists \*y'$ such that $g(\*y') \geq g(\*y^{*})$. Then we get $\*x' = \*V_{i}\Sigma_{i}^{-1}\*y'$ for which $f(\*x') \geq f(\*x^{*})$. This contradicts our assumption, unless $\*x' = \*x^{*}$. To maximize the score $\*x$ are chosen from the row space of $\*A_{i}$. Now without loss of generality we assume that $\norm{\*y} = 1$ and we know that if $\*x$ is in the row space of $\*A_{i}$ then $\*y$ is in the row space of $\*U_{i}$. Hence we get $\norm{\*U_{i}\*y} = \norm{\*y} = 1$.

We break denominator into sum of numerator and the rest, i.e. $\norm{\*U_{i}\*y}_{p}^{p} = \vert \tilde{\*u}_{i}^{T}\*y\vert^{p}+\sum_{j=1}^{i-1}\vert \tilde{\*u}_{j}^{T}\*y\vert^{p}$. % So we have $(\~E[\*a_{i}^{T}b])^{2} \leq \~E[(\*a_{i}^{T}b)^{2}] = \parallel b \parallel^{2}$, i.e. $\~E[\*a_{i}^{T}b] \leq \parallel b \parallel$. Now if $b = \*a_{i}$, then $\~E[\*a_{i}^{T}b] \leq \parallel b \parallel = \sqrt{d}$ and $\*a_{i}^{T}b = d$. The sensitivity score of row $i$ is $(sf(i))$ which is defined as follows,
% 
Consider the denominator term which is $\sum_{j=1}^{i-1}|\tilde{\*u}_{j}^{T}\*y|^{p} \geq \bigg(\sum_{j=1}^{i-1} |\tilde{\*u}_{j}^{T}\*y|^{2}\bigg)\cdot f(n)$. From this we estimate $f(n)$ as follows,
\begin{eqnarray}
\sum_{j=1}^{i-1}|\tilde{\*u}_{j}^{T}\*y|^{2} &=& \bigg(\sum_{j=1}^{i-1}|\tilde{\*u}_{j}^{T}\*y|^{2}\cdot 1\bigg) \nonumber \\
&\leq& \bigg(\sum_{j=1}^{i-1}|\tilde{\*u}_{j}^{T}\*y|^{2\cdot p/2}\bigg)^{2/p}\bigg(\sum_{j=1}^{i-1}1^{p/(p-2)}\bigg)^{1-2/p}\label{eqn:holder1}\\
&\leq& \bigg(\sum_{j=1}^{i-1}|\tilde{\*u}_{j}^{T}\*y|^{p}\bigg)^{2/p}\cdot(i)^{1-2/p} \label{eqn:last1}
\end{eqnarray}
% 
Here equation \eqref{eqn:holder1} is by holder's inequality, where $2/p + 1 - 2/p = 1$. So we rewrite the above term as $\big(\sum_{j=1}^{i-1}|\tilde{\*u}_{j}^{T}\*y|^{p}\big)^{2/p} \big(i\big)^{1-2/p} \geq \sum_{j=1}^{i-1} |\tilde{\*u}_{j}^{T}\*y|^{2} = 1 - |\tilde{\*u}_{i}^{T}\*y|^{2}$. Now substituting this in equation \eqref{eqn:last1} we get,
\begin{eqnarray*}
\bigg(\sum_{j=1}^{i-1}|\tilde{\*u}_{j}^{T}\*y|^{p}\bigg)^{2/p} &\geq& \bigg(\frac{1}{i}\bigg)^{1-2/p} (1 - |\tilde{\*u}_{i}^{T}\*y|^{2})\\
\bigg(\sum_{j=1}^{i-1}|\tilde{\*u}_{j}^{T}\*y|^{p}\bigg) &\geq& \bigg(\frac{1}{i}\bigg)^{p/2-1}(1 - |\tilde{\*u}_{i}^{T}\*y|^{2})^{p/2}
\end{eqnarray*}
So we get $\tilde{s}_{i} \leq \sup_{\*y}\frac{\vert \tilde{\*u}_{i}^{T}\*y\vert^{p}}{\vert \tilde{\*u}_{i}^{T}\*y\vert^{p}+(1/i)^{p/2-1}(1-\vert \tilde{\*u}_{i}^{T}\*y\vert^{2})^{p/2}}$. Note that this function increases with value of $|\tilde{\*u}_{i}^{T}\*y|$, which is maximum when $\*y = \frac{\tilde{\*u}_{i}}{\norm{\tilde{\*u}_{i}}}$,
\begin{equation}
 \tilde{s}_{i} \leq \frac{\norm{\tilde{\*u}_{i}}^{p}}{\norm{\tilde{\*u}_{i}}^{p} + (1/i)^{p/2-1}(1-\norm{\tilde{\*u}_{i}}^{2})^{p/2}} \nonumber
\end{equation}
We know that a function $\frac{a}{a+b} \leq \min\{1,a/b\}$, so we get $\tilde{l}_{i} = \min\{1,i^{p/2-1}\norm{\tilde{\*u}_{i}}^{p}\}$. Note that $\tilde{l}_{i} = i^{p/2-1}\norm{\tilde{\*u}_{i}}^{p}$ when $\norm{\tilde{\*u}_{i}}^{p} < (1/i)^{p/2-1}$ and it implies that $(1-\|\tilde{\*u}_{i}\|^{2})$ close to $1$.
% Now we consider 2 cases,
% \noindent\paragraph{Case 1:} Here we consider for rows with $\norm{\tilde{\*u}_{i}}^{p} < (1/i)^{p/2-1}$. Here each $S_{i} \leq i^{p/2-1}\norm{\tilde{\*u}_{i}}^{p}$. 
% \noindent\paragraph{Case 2:} Here we consider for rows with $\norm{\tilde{\*u}_{i}}^{p} \geq (1/i)^{2/p-1}$. Then each $S_{i} = 1$. 
% 
\end{proof}
% 
Here the scores are similar to leverage scores \cite{woodruff2014sketching} but due to $p$ order and data point coming in online manner the algorithm changes $i^{p/2-1}$ factor more for every row. Although we have bound on the $\sum_{i}^{n} \tilde{l}_{i}$ from lemma \ref{lemma:onlineSummationBound}, but this factor can be very huge as $i$ increases.
\subsubsection{Proof of Lemma \ref{lemma:onlineGuarantee}}
\begin{proof}{\label{proof:onlineGuarantee}}
 For simplicity we prove this lemma at the last timestamp $n$. But it can also be proved for any timestamp $t_{i}$ which is why the \online~can also be used in restricted streaming (online) setting.
 Now for a fixed $\*x \in \~R^{d}$ and its corresponding $\*y$, we define a random variables as follows, i.e. based on the choice of \online.
 \[ w_{i} =
  \begin{cases}
    \frac{1}{p_{i}}(\*u_{i}^{T}\*y)^{p}  & \quad \text{with probability } p_{i}\\
    0 & \quad \text{with probability } (1-p_{i})
  \end{cases}
\]
Where $\*u_{i}^{T}$ is the $i^{th}$ row of $\*U$ for $[\*U,\Sigma,\*V] = \mbox{svd}(\*A)$ and $\*y = \Sigma\*V^{T}\*x$. Here we get $\~E[w_{i}] = (\*u_{i}^{T}\*y)^{p}$. In our online algorithm we have defined $p_{i} = \min\{r\tilde{l}_{i}/\sum_{j=1}^{i}\tilde{l}_{j},1\}$. When $p_{i}$ is not $1$, we have $p_{i}= r\tilde{l}_{i}/\sum_{j=1}^{i}\tilde{l}_{j}\geq \frac{r|\*u_{i}^{T}\*y|^{p}}{\sum_{j=1}^{i}\tilde{l}_{j}\sum_{j=1}^{i}|\*u_{j}^{T}\*y|^{p}}\geq \frac{r|\*u_{i}^{T}\*y|^{p}}{\sum_{j=1}^{n}\tilde{l}_{j}\sum_{j=1}^{n}|\*u_{j}^{T}\*y|^{p}}$. Now to apply Bernstein \ref{thm:bernstein} we bound the term $\vert w_{i} - \~E[w_{i}]\vert \leq b$. Consider the two possible cases,
 
 \textbf{Case 1:} When $w_{i}$ is non zero, then $|w_{i} - \~E[w_{i}]| \leq |w_{i}| \leq \frac{|\*u_{i}^{T}\*y|^{p}}{p_{i}} \leq \frac{|\*u_{i}^{T}\*y|^{p}(\sum_{j=1}^{n} \tilde{l}_{j})\sum_{j=1}^{n} |\*u_{j}^{T}\*y|^{p}}{r|\*u_{i}^{T}\*y|^{p}} = \frac{(\sum_{j=1}^{n} \tilde{l}_{j})\sum_{j=1}^{n} |\*u_{j}^{T}\*y|^{p}}{r}$. Note for $p_{i}=1, \vert w_{i} - \~E[w_{i}]\vert = 0$.
 
 \textbf{Case 2} When $w_{i}$ is $0$ then $p_{i} < 1$. So we have $1 > \frac{r\tilde{l}_{i}}{\sum_{j=1}^{i}\tilde{l}_{j}} \geq \frac{r|\*u_{i}^{T}\*y|^{p}}{(\sum_{j=1}^{n} \tilde{l}_{j})\sum_{j=1}^{n}|\*u_{j}^{T}\*y|^{p}}$. So $\vert w_{i}-\~E[w_{i}]\vert = |\~E[w_{i}|] = |\*u_{i}^{T}\*y|^{p} < \frac{(\sum_{j=1}^{n} \tilde{l}_{j})\sum_{j=1}^{n} |\*u_{j}^{T}\*y|^{p}}{r}$. 
 
 So $b=\frac{(\sum_{j=1}^{n} \tilde{l}_{j})\sum_{j=1}^{n} |\*u_{j}^{T}\*y|^{p}}{r}$. Next we bound the variance of the sum $\sum_{i=1}^{n} \tilde{l}_{i}$. Let $\sigma^{2} = \mbox{var}\big(\sum_{i=1}^{n} w_{i}\big) = \sum_{i=1}^{n} \sigma_{i}^{2}$, where $\sigma_{i}^{2} = \mbox{var}(w_{i})$
% (Online) Further we bound the variance of the sum of random variables till $i$. Let $\mbox{var}_{i} = \mbox{var}\bigg(\sum_{j=1}^{i} w_{j}\bigg)$
 \begin{eqnarray}
  \sigma^{2} = \sum_{i=1}^{n} \sigma_{i}^{2} &=& \sum_{i=1}^{n} \~E[w_{i}^{2}] - (\~E[w_{i}])^{2} \nonumber \\
  &=& \sum_{i=1}^{n}\frac{|\*u_{i}^{T}\*y|^{2p}}{p_{i}} \nonumber \\
  &\leq& \sum_{i=1}^{n}\frac{|\*u_{i}^{T}\*y|^{2p}(\sum_{k=1}^{n} \tilde{l}_{k})\sum_{j=1}^{n} |\*u_{j}^{T}\*y|^{p}}{r|\*u_{i}^{T}\*y|^{p}} \nonumber \\
  &\leq& \frac{(\sum_{k=1}^{n} \tilde{l}_{k})(\sum_{j=1}^{n} |\*u_{j}^{T}\*y|^{p})^{2}}{r} \nonumber
 \end{eqnarray}
%  While first inequality is by definition of variance, the second equation is the expected value. The third inequality is for those $i$'s whose corresponding $p_{i}$'s are strictly less than $1$, as variance for $i$'s with $p_{i}=1$ would be $0$. 
 Note that $\|\*U\*y\|_{p}^{p} = \sum_{j=1}^{n} |\*u_{j}^{T}\*y|^{p}$, now setting $t = \epsilon \sum_{j=1}^{n} |\*u_{j}^{T}\*y|^{p}$, let 
 $$\~P = \mbox{Pr}\bigg(|W - \sum_{j=1}^{n} (\*u_{j}^{T}\*y)^{p}| \geq \epsilon \sum_{j=1}^{n} |\*u_{j}^{T}\*y|^{p}\bigg)$$
 \begin{eqnarray}
  \~P &\leq& \exp\bigg(\frac{\big(\epsilon \sum_{j=1}^{n} |\*u_{j}^{T}\*y|^{p}\big)^{2}}{2\sigma^{2}+bt/3}\bigg) \nonumber \\ 
%   &\leq& \exp\Bigg(\frac{-\epsilon^{2}(\|\*U\*y\|_{p}^{p})^{2}}{2\sum_{j=1}^{n}\tilde{l}_{j}(\|\*U\*y\|_{p}^{p})^{2}/r+ \epsilon\sum_{j=1}^{n}\tilde{l}_{j}(\|\*U\*y\|_{p}^{p})^{2}/3r} \Bigg) \nonumber\\
  &\leq& \exp\Bigg(\frac{-r\epsilon^{2}(\|\*U\*y\|_{p}^{p})^{2}}{(\|\*U\*y\|_{p}^{p})^{2}\sum_{j=1}^{n}\tilde{l}_{j}(2+\epsilon/3)}\Bigg) \nonumber \\
  &=& \exp\Bigg(\frac{-r\epsilon^{2}}{(2+\epsilon/3)\sum_{j=1}^{n}\tilde{l}_{j}}\Bigg) \nonumber
 \end{eqnarray}
 Now to ensure that the above probability at most $\delta$ we need to set $r = \frac{2k\sum_{j=1}^{n}\tilde{l}_{j}} {\epsilon^{2}} \log(\frac 1 \delta)$.
 
 Now to ensure the guarantee for $\ell_{p}$ subspace embedding one can define 
 \[ w_{i} =
  \begin{cases}
    \frac{1}{p_{i}}|\*u_{i}^{T}\*y|^{p}  & \quad \text{with probability } p_{i}\\
    0 & \quad \text{with probability } (1-p_{i})
  \end{cases}
 \]
and follow the above proof. By setting the $r = \frac{2k\sum_{j=1}^{n}\tilde{l}_{j}} {\epsilon^{2}} \log(\frac 1 \delta)$ one can get
$$\~P = \mbox{Pr}\bigg(|W - \|\*A\*x\|_{p}^{p}| \geq \epsilon \|\*A\*x\|_{p}^{p}\bigg) \leq \delta$$
One may follow the above proof to claim the final guarantee in equation \ref{eq:lp} using the same sampling complexity.
\end{proof}
% 
Now give the detail proof of sum of upper bounds of sensitivity scores.
\subsubsection{Proof of Lemma \ref{lemma:onlineSummationBound}}
\begin{proof}{\label{proof:onlineSummationBound}}
Recall that $\*A_i$ denotes the $i\times d$ matrix of the first $i$ vectors. \online~maintains the covariance matrix $\*M$. At the $(i-1)^{th}$ step we have $\*M = \*A_{i-1}^T\*A_{i-1}$. This is then used to define the score $\tilde{l}_{i}$  for the next step $i$, as $\tilde{l}_i = \min\{i^{p/2-1}\tilde{e}_{i}^{p/2},1\}$, where $\tilde{e}_i = \*a_{i}^T (\*A_{i}^T \*A_{i})^{\dagger} \*a_{i}$ and $\*a_{i}^{T}$ is the $i^{th}$ row. 
The scores $\tilde{e}_{i}$ are also called online leverage scores. We first give a bound on $\sum_{i=1}^{n} \tilde{e}_{i}$. A similar bound is given in the online matrix row sampling by~\cite{cohen2016online}, albeit for a regularized version of the scores $\tilde{e}_i$. 
As the rows are coming, the rank of $\*M$ increases 
from $1$ to at most $d$. We say that the algorithm is
in phase-$k$ if the rank of $\*M$ equals $k$. For each phase $k \in [1, d-1]$, let $i_k$ denote the index
where row $a_{i_k}$ caused a phase-change in $\*M$ i.e. 
rank of $(\*A_{i_k-1}^{T} \*A_{i_k-1})$ is $k-1$, while rank of $(\*A_{i_k}^{T} \*A_{i_k})$ is $k$. 
For each $i_k$, the leverage score $\tilde{e}_{i_k} = 1$, since row $\*A_{i_k}$
does not lie in the space of rows $\mathbf{a_1},\ldots, \mathbf{a_{i_k - 1}}$. There
are at most $d$ such indices $i_k$.

We now bound the $\sum_{i\in [i_k, i_{k+1}-1]} \tilde{e}_i$. Suppose the $\mbox{thin-SVD}(\*A_{i_k}^T \*A_{i_k})=\*V\mathbf{\Sigma_{i_k}} \*V^T$, all entries in $\mathbf{\Sigma_{i_k}}$ being positive. 
Furthermore, for any $i$ in this phase, i.e. for $i\in [i_{k}, i_{k+1}-1]$, $\*V$ forms the basis of the row space of $\*A_{i}$. Define
$\*X_{i} = \*V^T (\*A_{i}^T \*A_{i}) \*V$ and $\*a_{i} = \*V \*b_{i}$. Notice that each $\*X_{i}\in \Re^{k\times k}$, and $\*X_{i_k} = \mathbf{\Sigma_{i_k}}$. Also, $\*X_{i_k}$ is positive definite and hence full rank for each $i\in [i_{k}, i_{k+1}-1]$. We also have $\*X_{i} = \*X_{i-1} + \*b_{i} \*b_{i}^T$.

So we have, $\tilde{e}_{i} = \*a_{i}^T(\*A_{i}^T \*A_{i} )^{\dagger}\*a_{i} = \*b_{i}^T\*V^T(\*V(\mathbf{\Sigma_{i-1}}+\*b_{i} \*b_{i}^T)\*V^T)^{\dagger}\*V \*b_{i}=  \*b_{i}^T(\*X_{i-1}+\*b_{i}\*b_{i}^T)^{\dagger}\*b_{i} = \*b_{i}^T(\*X_{i-1}+\*b_{i}\*b_{i}^T)^{-1}\*b_{i}$
where the last equality uses the invertibility of the matrix. 
%  
% Note that matrix $\mathbf{\Sigma_{i-1}}+\*b_{i}\*b_{i}^T$ is a positive definite and hence a full rank matrix. 
 Now using matrix determinant lemma~\cite{vrabel2016note} on
 $\mbox{det}(\*X_{i-1}+\*b_{i}\*b_{i}^T)$ we get,
 \begin{eqnarray*}
  &=& \mbox{det}(\*X_{i-1})(1+\*b_{i}^T(\*X_{i-1})^{-1}\*b_{i}) \nonumber \\
  &\stackrel{(i)}{\geq}& \mbox{det}(\*X_{i-1})(1+\*b_{i}^T(\*X_{i-1}+\*b_{i} \*b_{i}^T)^{-1}\*b_{i}) \nonumber \\
  &=& \mbox{det}(\*X_{i-1})(1+\tilde{e}_{i}) \nonumber \\
  &\stackrel{(ii)}{\geq}& \mbox{det}(\*X_{i-1})\exp(\tilde{e}_{i}/2) \nonumber \\
  \exp(\tilde{e}_{i}/2) &\leq& \frac{\mbox{det}(\*X_{i-1}+\*b_{i}\*b_{i}^T)}{\mbox{det}(\*X_{i-1})} \nonumber
 \end{eqnarray*}
 Inequality $(i)$ follows as $\*X_{i-1}^{-1} - (\*X_{i-1} + \*b\*b^T)^{-1}\succeq 0$ (i.e. p.s.d.). Inequality $(ii)$ follows from
 the fact that $1+x \ge \exp(x/2)$ for $x \leq 1$ and by definition $\tilde{e}_{i} \leq 1$. Now since $\tilde{e}_{i_k} =1$, we analyze
%  \begin{align*}
%      \prod_{i\in[i_k+1, i_{k+1}-1]} \exp(\tilde{e}_{i}/2)  &\le  \prod_{i\in[i_k+1, i_{k+1}-1]}  \frac{\mbox{det}(X_{i})}{\mbox{det}(\*X_{i-1})} \le \frac{\mbox{det}(\mathbf{X_{i_{k+1}-1}})}{\mbox{det}(\*X_{i_k})}.
%  \end{align*}
 \begin{align*}
     \prod_{i\in[i_k+1, i_{k+1}-1]} \exp(\tilde{e}_{i}/2)  &\le  \prod_{i\in[i_k+1, i_{k+1}-1]}  \frac{\mbox{det}(\*X_{i})}{\mbox{det}(\*X_{i-1})}\\ &\le \frac{\mbox{det}(\mathbf{X_{i_{k+1}-1}})}{\mbox{det}(\mathbf{X_{i_k+1}})}.
 \end{align*}
 Taking the product over all the phases $\exp\bigg(\sum_{i\in [1, i_{d}-1]} \tilde{e}_{i}/2\bigg)$ gets,
%  \begin{align*}
%  \exp\bigg(\sum_{i\in [1, i_{d}-1]} \tilde{e}_{i}/2\bigg) & = \prod_{k\in [1,d-1]} \prod_{i\in[i_k+1, i_{k+1}-1]} \exp(\tilde{e}_{i}/2) \\
%  & =  \prod_{k\in [1,d-1]}\frac{\mbox{det}(\mathbf{X_{i_{k+1}-1}})}{\mbox{det}(\*X_{i_k})} =  \frac{\mbox{det}(\mathbf{X_{i_{d}-1}})}{\mbox{det}(X_{i_{1}})}\prod_{k\in [2,d-2]} \frac{\mbox{det}(\mathbf{X_{i_{k+1}-1}})}{\mbox{det}(X_{i_{k+1}})}
%  \end{align*}
\begin{align*}
 & = \exp((d-1)/2)\Big(\prod_{k\in [1,d-1]} \prod_{i\in[i_k+1, i_{k+1}-1]} \exp(\tilde{e}_{i}/2)\Big) \\
 & =  \exp((d-1)/2)\Big(\prod_{k\in [1,d-1]}\frac{\mbox{det}(\mathbf{X_{i_{k+1}-1}})}{\mbox{det}(\mathbf{X_{i_k+1}})}\Big)  \\ &=  \exp((d-1)/2)\Big(\frac{\mbox{det}(X_{i_{2}-1})}{\mbox{det}(X_{i_{1}})}\prod_{k\in [2,d-1]} \frac{\mbox{det}(\mathbf{X_{i_{k+1}-1}})}{\mbox{det}(\mathbf{X_{i_k+1}})}\Big)
 \end{align*}
 Because we know that $(\mathbf{A_{i_{k+1}-1}}^T \mathbf{A_{i_{k+1}-1}}) \succeq (\mathbf{A_{i_k+1}}^T \mathbf{A_{i_k+1}})$ so we get,
 $\mbox{det}(\mathbf{A_{i_{k+1}-1}}^T \mathbf{A_{i_{k+1}-1}}) \ge \mbox{det}(\mathbf{A_{i_k+1}}^T \mathbf{A_{i_k+1}})$. We get $\exp((d-1)/2)$ as there are $d-1$ many $i$ such that $\tilde{e}_{i} =1$. Hence, 
 \begin{align*}
     \exp\bigg(\sum_{i\in [1, i_{d}-1]} \tilde{e}_{i}/2\bigg) &\le \frac{\exp((d-1)/2)\mbox{det}(\mathbf{X_{i_{d}-1}})}{\mbox{det}(\mathbf{X_{i_1+1}})}\\ 
     &\le \frac{\exp((d-1)/2)\mbox{det}(\mathbf{A_{i_d}}^T \mathbf{A_{i_d}})}{\mbox{det}(\mathbf{X_{i_1+1}})}.
 \end{align*}
 Furthermore, we know $\tilde{e}_{i_d}=1$ so for $i\in [i_d, n]$, the matrix $\*M$ is full rank. We follow the same argument as above, and obtain
 \begin{align*}
 \exp\bigg(\sum_{i\in [i_d, n]} \tilde{e}_{i}/2\bigg) &\le \frac{\exp(1/2)\mbox{det}(\*A^T \*A)}{\mbox{det}(\mathbf{A_{i_d+1}}^T \mathbf{A_{i_d+1}})}\\
 & \le \frac{\exp(1/2)\|\*A\|^d}{\mbox{det}(\mathbf{A_{i_d+1}}^T \mathbf{A_{i_d+1}})} 
 \end{align*}
Let $\mathbf{a_{1}}$ be the first incoming row. Now multiplying the above two expressions and taking logarithm of both sides, and accounting for the indices $i_k$ for $k\in[2,d]$,
\begin{align*}
    \sum_{i\le n} \tilde{e}_i &\le d/2 + 2 d\log\|\*A\| - 2\log \|\*a_{1}\| \\
    &\le d/2 + 2 d\log\|\*A\| - \min_{i}2\log \|\*a_{i}\|. 
\end{align*}
Now, we give a bound on $\sum_{i=1}^{n} \tilde{l}_{i}$ where $\tilde{l}_{i} = \min\{1,i^{p/2-1}\tilde{e}_{i}^{p/2}\} \leq \min\{1,n^{p/2-1}\tilde{e}_{i}^{p/2}\}$. We consider two cases. When $\tilde{e}_i^{p/2} \geq n^{1-p/2}$ then $\tilde{l}_{i} = 1$, this implies that $\tilde{e}_i \geq n^{2/p-1}$. But we know $\sum_{i=1}^{n} \tilde{e}_i 
\le O(d+d\log\|\*A\|-\min_{i}\log \|\*a_{i})\|$ and hence there are at-most $O(n^{1-2/p}(d+d\log\|\*A\|-\min_{i}\log \|\*a_{i}\|))$ indices such $\tilde{l}_{i} = 1$. Now for the case where $\tilde{e}_{i}^{p/2} < n^{1-p/2}$, we get $\tilde{e}_{i}^{p/2-1} \leq (n)^{(1-p/2)(1-2/p)}$. Then $\sum_{i=1}^{n} n^{p/2-1}\tilde{e}_{i}^{p/2} = \sum_{i=1}^{n} n^{p/2-1}\tilde{e}_{i}^{p/2-1} \tilde{e}_{i} \leq \sum_{i=1}^{n}n^{1-2/p}\tilde{e}_{i}$ is $O(n^{1-2/p}(d+d\log\|\*A\|-\min_{i}\log \|\*a_{i}\|))$.
\end{proof}
% 
\subsection{\online+\mrlw}
As we know that any offline algorithm can be converted into a streaming algorithm by using merge and reduce method~\cite{har2004coresets}, so we apply merge and reduce on \cite{cohen2015p}. Now we discuss the guarantee that we get from the streaming version of \cite{cohen2015p}.
\subsubsection{Proof of Lemma \ref{lemma:Stream-MR}}
\begin{proof}\label{proof:Stream-MR}
 Here the data is coming in streaming sense and it is feed to the streaming version of the algorithm in \cite{cohen2015p} for $\ell_{p}$ subspace embedding. We use merge and reduce from \cite{har2004coresets} for streaming data. We call it \mrlw~. From \cite{cohen2015p} We know that for a set $\*P$ of size $n$ takes $O(nd^{p/2})$ time to return a coreset $\*Q$ of size $O(d^{p/2}(\log d)(\log 1/\epsilon)\epsilon^{-5})$. Note that for the \mrlw~in section 7 of \cite{har2004coresets} we set $M=O(d^{p/2}(\log d)(\log 1/\epsilon)\epsilon^{-5})$. The method returns $\*Q_{i}$ as the $(1 + \delta_{i})$ coreset for the partition $\*P_{i}$ where $|\*P_{i}|$ is either $2^{i}M$ or $0$, here $\rho_{j} = \epsilon/(c(j+1)^{2})$ such that $1+\delta_{i} = \prod_{j=0}^{i} (1 + \rho_{j}) \leq 1 + \epsilon/2, \forall j \in \lceil \log n \rceil$. Thus we have $|\*Q_{i}|$ is $O(d^{p/2}(\log d)(\log 1/\epsilon)(i+1)^{10}\epsilon^{-5})$. In \mrlw~the method reduce sees at max $\log n$ many coresets at any point of time. Hence the total working space is $O(d^{p/2}(\log^{11} n)(\log d)(\log 1/\epsilon)\epsilon^{-5})$. Now the amortized time spent per update is,
 \begin{eqnarray*}
  && \sum_{i=1}^{\lceil \log (n/M) \rceil} \frac{1}{2^{i}M}(|\*Q_{i}|d^{p/2}) \\
  &=& \sum_{i=1}^{\lceil \log (n/M) \rceil} \frac{1}{2^{i}M}(M(i+1)^{4}d^{p/2}) \leq d^{p/2}
 \end{eqnarray*}
So the finally the algorithm return $\*Q$ as the final coreset of $O(d^{p/2}(\log^{10} n)(\log d)(\log 1/\epsilon)\epsilon^{-5})$ rows and uses $O(d^{p/2})$ ammortized update time.
\end{proof}
% 
Next we discuss the guarantee that we get by feeding the output of \online~to \mrlw~of \cite{cohen2015p}
\subsubsection{Proof of Lemma \ref{thm:improvedStream-MR}}
\begin{proof}\label{proof:improvedStream-MR}
 Here the data is coming in streaming sense. Now \online~filters out the rows with small sensitivity scores and only the sampled rows (high sensitivity score) are feed to \mrlw. We know that for a set $\*P$ of size $n$ takes $O(nd^{p/2})$ time to return a coreset $\*Q$ of size $O(d^{p/2}(\log d)(\log 1/\epsilon)\epsilon^{-5})$ by \cite{cohen2015p}. But here the \online~ensures that \mrlw~only gets $\tilde{O}(n^{1-2/p}d)$, hence the amortized update time is same as that of \online, i.e. $O(d^{2})$. Now similar to the above proof \ref{proof:improvedStream-MR}, by the \mrlw~from section 7 of \cite{har2004coresets} we set $M=O(d^{p/2}(\log d)(\log 1/\epsilon)\epsilon^{-5})$. The method returns $\*Q_{i}$ as the $(1 + \delta_{i})$ coreset for the partition $\*P_{i}$ where $|\*P_{i}|$ is either $2^{i}M$ or $0$, here $\rho_{j} = \epsilon/(c(j+1)^{2})$ such that $1+\delta_{i} = \prod_{j=0}^{i} (1 + \rho_{j}) \leq 1 + \epsilon/2, \forall j \in \lceil \log n \rceil$. Thus we have $|\*Q_{i}|$ is $O(d^{p/2}(\log d)(\log 1/\epsilon)(i+1)^{10}\epsilon^{-5})$. Hence the total working space is $O((1-2/p)^{11}d^{p/2}(\log^{11} n)(\log d)(\log 1/\epsilon)\epsilon^{-5})$. So finally \online+\mrlw~returns a coreset $\*Q$ of $O((1-2/p)^{10}d^{p/2}(\log^{10} n)(\log d)(\log 1/\epsilon)\epsilon^{-5})$ rows.
\end{proof}
% 
\subsection{\kernelfilter}
In this section we discuss the supporting lemma for proving the theorem~\ref{thm:slowOnline}. 
\subsection{Proof of Lemma~\ref{lemma:kernel}}
\begin{proof}{\label{proof:kernel}}
The term $|\*x^{T}\*y|^{p}=|\*x^{T}\*y|^{\lfloor p/2 \rfloor}|\*x^{T}\*y|^{\lceil p/2 \rceil}$. We define $|\*x^{T}\*y|^{\lfloor p/2 \rfloor} = |\grave{\*a}_{i}^{T}\grave{\*x}|$ and $|\*x^{T}\*y|^{\lceil p/2 \rceil} = |\acute{\*a}_{i}^{T}\acute{\*x}|$. For even valued $p$ we know $\lfloor p/2 \rfloor=\lceil p/2 \rceil$, so for simplicity we write as $|\*x^{T}\*y|^{p/2} = |\hat{\*a}_{i}^{T}\hat{\*x}|$. So we get $|\*x^{T}\*y|^{p} = |\langle \*x \otimes^{p/2}, \*y \otimes^{p/2} \rangle|^{2} = |\hat{\*x}^{T}\hat{\*y}|^{2}$. Here the vector $\hat{\*x} = \mbox{vec}(\*x \otimes^{p/2}) \in \~R^{p/2}$ and similarly $\hat{\*y}$ is also defined. Now for odd value of $p$ we have $\grave{\*x} = \mbox{vec}(\*x \otimes^{(p-1)/2}) \in \~R^{(p-1)/2}$ and $\acute{\*x} = \mbox{vec}(\*x \otimes^{(p+1)/2}) \in \~R^{(p+1)/2}$. Similarly $\grave{\*y}$ and $\acute{\*y}$ are defined for odd value of $p$. So we get $|\*x^{T}\*y|^{p} = |\langle \*x \otimes^{(p-1)/2}, \*y \otimes^{(p-1)/2} \rangle|\cdot|\langle \*x \otimes^{(p+1)/2}, \*y \otimes^{(p+1)/2} \rangle| = |\grave{\*x}^{T}\grave{\*y}|\cdot|\acute{\*x}^{T}\acute{\*y}|$.
\[|\*x^{T}\*y|^{p} =
  \begin{cases}
  |\hat{\*x}^{T}\hat{\*y}|^{2}  & \quad \text{if } p \mbox{ even}\\
  |\grave{\*x}^{T}\grave{\*y}|\cdot|\acute{\*x}^{T}\acute{\*y}|  & \quad \text{if } p\ \mbox{odd}\\
  \end{cases}
\]
\end{proof}
% 
\subsubsection{Proof of Lemma~\ref{lemma:slowOnlineSensitivityBound}}
\begin{proof}{\label{proof:slowOnlineSensitivityBound}}
 We define the online sensitivity scores $\tilde{s}_{i}$ for each point $i$ as follows,
  \allowdisplaybreaks
  \begin{eqnarray*}
  s_{i} &=& \sup_{\{\*x\mid\|\*x\|=1\}}\frac{|\*a_{i}^{T}\*x|^{p}}{\|\*A_{i}\*x\|_{p}^{p}}\\
  &=&\sup_{\{\*x\mid\|\*x\|=1\}}\frac{|\*a_{i}^{T}\*x|^{\lfloor p/2 \rfloor}|\*a_{i}^{T}\*x|^{\lceil p/2 \rceil}}{\|\*A_{i}\*x\|_{p}^{p}} \\
  &=& \sup_{\{\*x,\grave{\*x},\acute{\*x}\mid\|\*x\|=1\}}\frac{|\grave{\*x}^{T}\grave{\*a}_{i}\acute{\*a}_{i}^{T}\acute{\*x}|}{\|\*A_{i}\*x\|_{p}^{p}} \\
  &\stackrel{(i)}=& \sup_{\{\*x,\grave{\*z},\acute{\*z}\mid\|\*x\|=1\}}\frac{|\grave{\*z}^{T}\grave{\*u}_{i}\acute{\*u}_{i}^{T}\acute{\*z}|\cdot(\|\grave{\*z}\|\|\acute{\*z}\|)}{\|\*A_{i}\*x\|_{p}^{p}\cdot(\|\grave{\*z}\|\|\acute{\*z}\|)} \\
  &\stackrel{(ii)}\leq& \sup_{\{\*x,\grave{\*y},\acute{\*y},\grave{\*z},\acute{\*z}\mid\|\*x\|=1\}}\frac{|\grave{\*y}^{T}\grave{\*u}_{i}\acute{\*u}_{i}^{T}\acute{\*y}|\cdot(\|\grave{\*z}\|\|\acute{\*z}\|)}{\|\*A_{i}\*x\|_{p}^{p}} \\
  &\stackrel{(iii)}=& \sup_{\{\*x,\grave{\*y},\acute{\*y},\grave{\*z},\acute{\*z}\mid\|\*x\|=1\}}\frac{|\grave{\*y}^{T}\grave{\*u}_{i}\acute{\*u}_{i}^{T}\acute{\*y}|\cdot\|\grave{\*z}\acute{\*z}^{T}\|}{\|\*A_{i}\*x\|_{p}^{p}} \\
  &\stackrel{(iv)}=& \sup_{\{\*x,\grave{\*y},\acute{\*y}\mid\|\*x\|=1\}}\frac{|\grave{\*y}^{T}\grave{\*u}_{i}\acute{\*u}_{i}^{T}\acute{\*y}|\cdot\|\grave{\Sigma}_{i}\grave{\*V}_{i}^{T}\grave{\*x}\acute{\*x}^{T}\acute{\*V}_{i}\acute{\Sigma}_{i}\|}{\|\*A_{i}\*x\|_{p}^{p}} \\
  &\stackrel{(v)}=& \sup_{\{\*x,\grave{\*y},\acute{\*y}\mid\|\*x\|=1\}}\frac{|\grave{\*y}^{T}\grave{\*u}_{i}\acute{\*u}_{i}^{T}\acute{\*y}| \cdot\|\grave{\*U}_{i}\grave{\Sigma}_{i}\grave{\*V}_{i}^{T}\grave{\*x}\acute{\*x}^{T}\acute{\*V}_{i}\acute{\Sigma}_{i}\acute{\*U}_{i}^{T}\|}{\|\*A_{i}\*x\|_{p}^{p}} \\
  &=& \sup_{\{\*x,\grave{\*y},\acute{\*y}\mid\|\*x\|=1\}}\frac{|\grave{\*y}^{T}\grave{\*u}_{i}\acute{\*u}_{i}^{T}\acute{\*y}|\cdot\|\grave{\*A}_{i}\grave{\*x}\acute{\*x}^{T}\acute{\*A}_{i}^{T}\|}{\|\*A_{i}\*x\|_{p}^{p}} \\
  &\stackrel{(vi)}=& \sup_{\{\*x,\grave{\*y},\acute{\*y}\mid\|\*x\|=1\}}\frac{|\grave{\*y}^{T}\grave{\*u}_{i}\acute{\*u}_{i}^{T}\acute{\*y}|\cdot\mbox{Trace}(\grave{\*A}_{i}\grave{\*x}\acute{\*x}^{T}\acute{\*A}_{i}^{T})}{\|\*A_{i}\*x\|_{p}^{p}} \\
  &\leq& \sup_{\{\*x,\grave{\*y},\acute{\*y}\mid\|\*x\|=1\}}\frac{|\grave{\*y}^{T}\grave{\*u}_{i}\acute{\*u}_{i}^{T}\acute{\*y}|\cdot\big(\sum_{j=1}^{i}|\grave{\*x}^{T}\grave{\*a}_{j}\acute{\*a}_{j}^{T}\acute{\*x}|\big)}{\|\*A_{i}\*x\|_{p}^{p}} \\
  &=& \sup_{\{\*x,\grave{\*y},\acute{\*y}\mid\|\*x\|=1\}}\frac{|\grave{\*y}^{T}\grave{\*u}_{i}\acute{\*u}_{i}^{T}\acute{\*y}|\cdot\|\*A_{i}\*x\|_{p}^{p}}{\|\*A_{i}\*x\|_{p}^{p}} \\
  &=& \sup_{\{\grave{\*y},\acute{\*y}\mid\|\*x\|=1\}}|\grave{\*y}^{T}\grave{\*u}_{i}\acute{\*u}_{i}^{T}\acute{\*y}| \\
  &=& \|\grave{\*u}_{i}\|\cdot\|\acute{\*u}_{i}\|
 \end{eqnarray*}
 Let $\grave{\*A}$ be the matrix where its $j^{th}$ row $\grave{\*a}_{j} = \*a \otimes^{d^{\lfloor p/2 \rfloor}} \in \~R^{d^{\lfloor p/2 \rfloor}}$ and $\acute{\*A}$ be the matrix where its $j^{th}$ row $\acute{\*a}_{j} = \*a \otimes^{d^{\lceil p/2 \rceil}} \in \~R^{d^{\lceil p/2 \rceil}}$. Further let $\grave{\*A}_{i}$ and $\acute{\*A}_{i}$ are the corresponding matrices $\*A_{i} \in \~R^{i \times d}$ which represents first $i$ streaming rows. We define $[\grave{\*U}_{i},\grave{\Sigma}_{i},\grave{\*V}_{i}] = \mbox{svd}(\grave{\*A}_{i})$ such that $\grave{\*a}_{i}^{T} = \grave{\*u}_{i}^{T}\grave{\Sigma}_{i}\grave{\*V}_{i}^{T}$ and $[\acute{\*U}_{i},\acute{\Sigma}_{i},\acute{\*V}_{i}] = \mbox{svd}(\acute{\*A}_{i})$ such that $\acute{\*a}_{i}^{T} = \acute{\*u}_{i}^{T}\acute{\Sigma}_{i}\acute{\*V}_{i}^{T}$. Now for a fixed $\*x \in \~R^{d}$ its corresponding $\grave{\*x}$ and $\acute{\*x}$ are also fixed in their corresponding higher dimensions. Here $\grave{\Sigma}_{i}\grave{\*V}_{i}^{T}\grave{\*x}=\grave{\*z}$ and $\acute{\Sigma}_{i}\acute{\*V}_{i}^{T}\acute{\*x}=\acute{\*z}$ from which we define unit vectors $\grave{\*y} = \grave{\*z}/\|\grave{\*y}\|$ and $\acute{\*y} = \acute{\*z}/\|\acute{\*z}\|$.

 Now The equality (i) is by change of variable where $\grave{\*z}$ and $\acute{\*z}$ are still function of $\*x$ so $\sup$ is still over $\*x$. The (ii) inequality is due to the $\sup$ is over $\grave{\*y}$ and $\acute{\*y}$. Note that $\forall \*x, \exists \grave{\*y}$ and $\exists \acute{\*y}$. The (iii) equality follows by the fact that for any two vector $\*a$ and $\*b$ we know that $\|\*a\|\|\*b\| = \|\*a\*b^{T}\|_{F} = \|\*a\*b^{T}\|$. The (iv) equality is directly by substituting $\grave{\*z}$ and $\acute{\*z}$. The (v) equality follows by adding scale invariant orthonormal column basis $\grave{\*U}$ and $\acute{\*U}$. The (vi) is equality is by the fact that the matrix $\grave{\*A}\grave{\*x}\acute{\*x}^{T}\acute{\*A}^{T}$ is a rank-1 matrix, so we know that the spectral norm is equal to the Frobenius norm of the matrix which is further equal to the trace of the matrix. 

 Note that for even value $p$ we get $\tilde{s}_{i} \leq \|\hat{\*u}_{i}\|^{2}$ as by lemma \ref{lemma:kernel} we get $\grave{\*u}_{i} = \acute{\*u}_{i} = \hat{\*u}_{i}$. In fact this is true because for even $p$ we have $\lfloor p/2 \rfloor = \lceil p/2 \rceil$.
\end{proof}
% 
\subsubsection{Proof of Lemma \ref{lemma:slowOnlineGuarantee}}
\begin{proof}{\label{proof:slowOnlineGuarantee}}
 For simplicity we prove this lemma at the last timestamp $n$. But it can also be proved for any timestamp $t_{i}$ which is why the \kernelfilter~can also be used in restricted streaming (online) setting. Also for a change we show this for $\ell_{p}$ subspace embedding. Now for some fixed $\*x \in \~R^{d}$ we get $\grave{\*x}$ and $\acute{\*x}$. Our algorithm takes the following random variable for every row $i$.
\[ w_{i} =
  \begin{cases}
    (1/p_{i}-1)|\grave{\*x}^{T}\grave{\*a}_{i}\acute{\*a}_{i}^{T}\acute{\*x}|  & \quad \text{w.p. } p_{i}\\
    -|\grave{\*x}^{T}\grave{\*a}_{i}\acute{\*a}_{i}^{T}\acute{\*x}| & \quad \text{w.p. } (1-p_{i})
  \end{cases}
\]
% 
 Now to show the concentration of the expected term we will apply Bernstein's inequality on $W = \sum_{i=1}^{n} w_{i}$. For this first we bound $|w_{i} - \~E[w_{i}]| = |w_{i}|$ as $\~E[w_{i}] = 0$ and then we give a bound on $\mbox{var}(W)$. 

 Now for the $i^{th}$ timestamp if $p_{i}=1$ then $|w_{i}| = 0$, else if $p_{i} <1$ and \kernelfilter~samples the row then $|w_{i}| \leq |\grave{\*x}^{T}\grave{\*a}_{i}\acute{\*a}_{i}^{T}\acute{\*x}|/p_{i} = \|\grave{\*z}\|\|\acute{\*z}\||\grave{\*y}^{T}\grave{\*u}_{i}\acute{\*u}_{i}^{T}\acute{\*y}|/p_{i} \leq \|\grave{\*z}\|\|\acute{\*z}\|\|\grave{\*u}_{i}\|\|\acute{\*u}_{i}\|/(r \|\grave{\*u}_{i}\|\|\acute{\*u}_{i}\|) = \|\grave{\*z}\|\|\acute{\*z}\|/r$. Next when \kernelfilter~does not sample means that $p_{i} < 1$, then $1 > r\|\grave{\*u}_{i}\|\|\acute{\*u}_{i}\| \geq r|\grave{\*y}^{T}\grave{\*u}_{i}\acute{\*u}_{i}^{T}\acute{\*y}|=r|\grave{\*x}^{T}\grave{\*a}_{i}\acute{\*a}_{i}^{T}\acute{\*x}|/ (\|\grave{\*z}\|\|\acute{\*z}\|)$. Finally we get $|\grave{\*x}^{T}\grave{\*a}_{i}\acute{\*a}_{i}^{T}\acute{\*x}| \leq \|\grave{\*z}\|\|\acute{\*z}\|/r$. So for each $i$ we get $|w_{i}| \leq \|\grave{\*z}\|\|\acute{\*z}\|/r \leq \|\*A\*x\|_{p}^{p}$. It is true from the argument analysis \ref{proof:slowOnlineSensitivityBound}.

Next we bound $\sigma^{2} = \mbox{var}(W) = \sum_{i=1}^{n} \mbox{var}(w_{i}) = \sum_{i=1}^{n}\~E[w_{i}^{2}]$ as follows,

\begin{eqnarray*}
 \sigma^{2} &=& \mbox{var}(W) = \sum_{i=1}^{n} \~E[r_{i}^{2}] \\
 &=& \sum_{i=1}^{n} |\grave{\*x}^{T}\grave{\*a}_{i}\acute{\*a}_{i}^{T}\acute{\*x}|^{2}/p_{i} \\
 &=& \sum_{i=1}^{n} \|\grave{\*z}\|\|\acute{\*z}\||\grave{\*y}^{T}\grave{\*u}_{i}\acute{\*u}_{i}^{T}\acute{\*y}||\grave{\*x}^{T}\grave{\*a}_{i}\acute{\*a}_{i}^{T}\acute{\*x}|/p_{i} \\
 &\leq& \|\grave{\*z}\|\|\acute{\*z}\| \|\*A\*x\|_{p}^{p}/r \\
 &\leq& \|\*A\*x\|_{p}^{2p}/r 
%  &\leq& ((\sigma^{\max}_{(p-1)})^{(p-1)}(\sigma^{\max}_{(p+1)})^{(p+1)})^{1/2}\|\*A\*x\|_{p}^{p}/c
\end{eqnarray*}
% 
We obtain the last inequality by similar analysis in \ref{proof:slowOnlineSensitivityBound}. Now we can apply Bernstein \ref{thm:bernstein} on the sum of random variables to bound the event $\~P: \mbox{Pr}(|W| \geq \epsilon\|\*A\*x\|_{p}^{p})$, Here we have $b = \|\*A\*x\|_{p}^{p}/r, \sigma^{2} = \|\*A\*x\|_{p}^{2p}/r$ and we set $t = \epsilon\|\*A\*x\|_{p}^{p}$, then we get
\begin{eqnarray*}
 \~P &\leq& \exp\bigg(\frac{-(\epsilon\|\*A\*x\|_{p}^{p})^{2}}{2\|\*A\*x\|_{p}^{2p}/r+\epsilon\|\*A\*x\|_{p}^{2p}/3r}\bigg) \\
 &=& \exp\bigg(\frac{-r\epsilon^{2}\|\*A\*x\|_{p}^{2p}}{(2+\epsilon/3)\|\*A\*x\|_{p}^{2p}}\bigg) \\
 &=& \exp\bigg(\frac{-r\epsilon^{2}}{(2+\epsilon/3)}\bigg)
\end{eqnarray*}
To ensure the above event with probability at least $1-\delta$ we need to set $r \geq \log(1/\delta)/\epsilon^{-2}$. 

 Now to ensure the guarantee for tensor contraction as equation \eqref{eq:contract} one can define 
 \[ w_{i} =
  \begin{cases}
   (1/p_{i}-1)(\grave{\*x}^{T}\grave{\*a}_{i}\acute{\*a}_{i}^{T}\acute{\*x})  & \quad \text{w.p. } p_{i}\\
   -(\grave{\*x}^{T}\grave{\*a}_{i}\acute{\*a}_{i}^{T}\acute{\*x}) & \quad \text{w.p. } (1-p_{i})
  \end{cases}
 \]
and follow the above proof. By setting the $r = \frac{2k\sum_{j=1}^{n}\tilde{l}_{j}} {\epsilon^{2}} \log(\frac 1 \delta)$ one can get
$$\~P = \mbox{Pr}\bigg(|W - \sum_{j=1}^{n} (\*u_{j}^{T}\*y)^{p}| \geq \epsilon \sum_{j=1}^{n} |\*u_{j}^{T}\*y|^{p}\bigg) \leq \delta$$
One may follow the above proof to claim the final guarantee in equation \ref{eq:contract} using the same sampling complexity.
\end{proof}
% 
\subsubsection{Proof of Lemma \ref{lemma:slowOnlineSummationBound}}
\begin{proof}{\label{proof:slowOnlineSummationBound}}
 Using proof of lemma \ref{lemma:onlineSummationBound}, let $\grave{c}_{i} = \|\grave{\*u}_{i}\|^{2}$ and $\acute{c}_{i} = \|\acute{\*u}_{i}\|^{2}$. Now $\sum_{i=1}^{n} \tilde{l}_{i}\leq\sum_{i=1}^{n}\grave{c}_{i} + \acute{c}_{i}$. From lemma \ref{lemma:onlineSummationBound} we get $\sum_{i=1}^{n} \grave{c}_{i}$ is $O(d^{\lfloor p/2 \rfloor}(1+\log \|\grave{\*A}\|)-\min_{i}\log \|\grave{\*a}_{i}\|)$. Now with $[\*u,\Sigma,\*V] = \mbox{svd}(\*A)$ we have $\grave{\*a}^{T} = \mbox{vec}(\*a_{i}^{T} \otimes^{\lfloor p/2 \rfloor}) = \mbox{vec}((\*u_{i}^{T}\Sigma\*V^{T})^{\lfloor p/2 \rfloor})$. So we get $\|\grave{\*A}\| \leq \sigma_{1}^{\lfloor p/2 \rfloor}$. Hence $\sum_{i=1}^{n}$ is $O(d^{\lfloor p/2 \rfloor}(1+\lfloor p/2 \rfloor\log \|\*A\|)-\lfloor p/2 \rfloor\min_{i}\log \|\*a_{i}\|)$. Similarly we get $\sum_{i=1}^{n} \acute{c}_{i}$ is $O(d^{\lceil p/2 \rceil}(1+\lceil p/2 \rceil\log \|\*A\|)-\lceil p/2 \rceil\min_{i}\log \|\*a_{i}\|)$. So finally $\sum_{i=1}^{n} \tilde{l}_{i}$ is $O(d^{\lceil p/2 \rceil}(1+\lceil p/2 \rceil\log \|\*A\|)-\lfloor p/2 \rfloor\min_{i}\log \|\*a_{i}\|)$.
\end{proof}
% 
\subsection{$p=2$ case}{\label{app:matrix}}
In the following theorem we state the guarantees of our algorithm in the matrix case. 
\begin{corollary}
\label{lem:matrixcoreset}
 Given a matrix $\*A \in \~R^{n\times d}$ with rows coming one at a time, for $p=2$ the algorithm \ref{alg:onlineCoreset} takes $O(d^{2})$ update time and samples $O(\frac{d}{\epsilon^{2}}(d+d\log\|\*A\|-\min_{i}\log \|\*a_{i}\|))$ rows and preserves the following with probability at least 0.9, $\forall \*x \in \~R^{d}$
 $(1-\epsilon)\|\*A\*x\|^{2} \leq \|\*C\*x\|^{2} \leq (1+\epsilon)\|\*A\*x\|^{2}$.
\end{corollary}
% 
Just by using Matrix Bernstein inequality~\cite{tropp2011freedman} we can slightly improve the sampling complexity from $O(d^{2})$ to $O(d\log d)$. For simplicity we modify the sampling probability to $p_{i} = \min\{r\tilde{l}_{i},1\}$ and get the following guarantee.
\begin{theorem}{\label{thm:improvedMatrixCoreset}}
 The above modified algorithm samples $O(\frac{\log d}{\epsilon^{2}}(d+d\log\|\*A\|-\min_{i} \log \|\*a_{i}\|))$ rows and preserves the following with probability at least 0.9, $\forall \*x \in \~R^{d}$
 \begin{align*}
  (1-\epsilon)\|\*A\*x\|^{2} \leq \|\*C\*x\|^{2} \leq (1+\epsilon)\|\*A\*x\|^{2}
 \end{align*}
\end{theorem}
\begin{proof}{\label{proof:improvedMatrixCoreset}}
We prove this lemma in 2 parts. First we show that sampling $\*a_{i}$ with probability $p_{i}=\min\{r\tilde{l}_{i},1\}$ where $\tilde{l}_{i} = \min\{(1+\epsilon)\*a_{i}^{T}(\*A_{i}^{T}\*A_{i})^{\dagger}\*a_{i},1\}$ preserves $\|\*C^{T}\*C\| \leq (1\pm \epsilon)\|\*A^{T}\*A\|, \forall \*x \in \~R^{d}$. Next we give the bound on expected sample size.

We define, $u_{i} = (\*A^{T}\*A)^{-1/2}\*a_{i}$ and we define a random matrix $\*X_{i}$ corresponding to each streaming row as, 
\[ \*X_{i} =
  \begin{cases}
    (1/p_{i} - 1)\*u_{i}\*u_{i}^{T}  & \quad \text{if } \*a_{i} \text{ is sampled in } \tilde{A}\\
    -\*u_{i}\*u_{i}^{T} & \quad \text{else}
  \end{cases}
\]
Now we have,
\begin{eqnarray*}
 \tilde{l}_{i} &=& \*a_{i}^{T}(\*A_{i-1}^{T}\*A_{i-1}+\*a_{i}\*a_{i}^{T})^{\dagger}\*a_{i}\\
 &\geq& \*a_{i}^{T}(\*A^{T}\*A)^{\dagger}\*a_{i}\\
 &=& \*u_{i}^{T}\*u_{i}
\end{eqnarray*}
For $p_{i} \geq \min\{r\*u_{i}^{T}\*u_{i},1\}$, if $p_{i} = 1$, then $\|\*X_{i}\| = 0$, else $p_{i} = r\*u_{i}^{T}\*u_{i} < 1$. So we get $\norm{\*X_{i}} \leq 1/r$. 
Next we bound $\~E[\|\*X_{i}\|^{2}]$.
\begin{eqnarray*}
 \~E[\|\*X_{i}\|^{2}] &=& p_{i}(1/p_{i}-1)^{2}\|\*u_{i}\*u_{i}^{T}\|^{2}+(1-p_{i})\|\*u_{i}\*u_{i}^{T}\|^{2} \\
 &\preceq& \|\*u_{i}\*u_{i}^{T}\|^{2}/p_{i} \\
 &\preceq& \|\*u_{i}\*u_{i}^{T}\|/r
\end{eqnarray*}
Let $\*X = \sum_{i=1}^{n} \*X_{i}$, then variance of $\|\*X\|$ 
\begin{align*}
\mbox{var}(\norm{\*X}) &=& \sum_{i=1}^{n}\mbox{var}(\|\*X_{i}\|) \leq \sum_{i=1}^{n} \~E[\|\*X_{i}\|^{2}] \\
&\leq& \bigg\lVert\sum_{j=1}^{n} \*u_{j}\*u_{j}^{T}/r \bigg\rVert \leq 1/r
\end{align*}
Next by applying matrix Bernstein theorem \ref{thm:matrixBernstein} with appropriate $r$ we get,
$$\mbox{Pr}(\norm{\*X}\geq\epsilon) \leq d \exp\bigg(\frac{-\epsilon^{2}/2}{1/r+\epsilon/(3r)}\bigg)\leq 1/d$$
This implies that the modified algorithm preserves spectral approximation with high probability, i.e. $\norm{\*C^{T}\*C} \leq (1 \pm \epsilon)\norm{\*A^{T}\*A}$.

Now once we have the above result using lemma 4 of \cite{cohen2015uniform} we conclude that the expected number of samples are $O(\sum_{i=1}^{n}\tilde{l}_{i}(\log d)/\epsilon^{2})$. Now from lemma \ref{lemma:onlineSummationBound} we know that for $p=2, \sum_{i=1}^{n}\tilde{l}_{i}$ is $O(d(1+2\log \|\*A\|))$. Finally to get $\mbox{Pr}(\norm{\*X}\geq\epsilon) \leq \delta$ the algorithm samples $O(\frac{d\log d}{\epsilon^{2}}(1+2\log\|\*A\|)\log(1/\delta)$ rows.
\end{proof}
% 
\subsection{Latent Variable Modeling}{\label{sec:application}}
Under the assumption that the data is generated by some generative model such as Gaussian Mixture model, Topic model, Hidden Markov model etc, one can represent the data in terms of higher order moments to realize the latent variables~\cite{anandkumar2014tensor}. Next these higher order moments (Tensors) are reduced to smaller dimensional tensors, which are also orthogonally decomposable. This process is called whitening. For example 
Rephrasing the main theorem 5.1~\cite{anandkumar2014tensor} we get that the $\|\mcal M_{3} - \widetilde{\mcal M}_{3}\| \leq \varepsilon\|\*W\|^{-3}$ where $\mcal M_{3}$ is the true tensor and $\widetilde{\mcal M}_{3}$ is the empirical tensor.
Now we state the guarantees that one gets by applying the RTPI on our sampled data.
\begin{corollary}{\label{coro:tensorFactors}}
 For a dataset with rows coming in streaming fashion and the algorithm \online+\kernelfilter~guarantees \eqref{eq:contract} such that for all unit vector $\*x \in \*Q$, it ensures $\epsilon\sum_{i \leq n} |\*a^{T}\*x|^{3} \leq \varepsilon \|\*W\|^{-3}$. Then applying the RTPI on the sampled coreset $\*C$ returns $k$ eigenpairs $\{\lambda_{i},\*v_{i}\}$ of the reduced (orthogonally decomposable) tensor, ensures that $\forall i \in [k]$,
 $$\|\*v_{\pi(i)} - \*v_{i}\| \leq 8\varepsilon/\lambda_{i} \qquad \qquad |\lambda_{\pi(i)}-\lambda_{i}| \leq 5\varepsilon$$
\end{corollary}
In the appendix we show empirically that how coreset from \online+\kernelfilter~can preserve tensor contraction. We further show that the coresets the can also used We also show our algorithm's performance for online single topic modeling application. We compare our method with 2 other sampling schemes, namely -- uniform and leverage score. 


We also ran single topic modeling on synthetic data. We defined 12 topic vectors in $\mathbb{R}^{30}$, where we ensured that one of the vector is orthogonal to rest of the vectors. Next we generated 10k documents based on these topic distributions. First we sampled a topic and based on its distribution we generated few words and called that a document. Number of words in each documents were decided uniformly at random between 3 and 60. We also ensured that a topic  remained orthogonal to the rest of the topics and generated at most 20 documents. We ran our sampling algorithm and then ran tensor power iteration for topic modeling as given in \cite{anandkumar2014tensor}. Next we matched the true topic distribution with the estimated topic distribution and here we report the sum of $\ell_{1}$ distance of difference between true and matched estimated topic.

% \begin{table}[htbp]
%     \centering
%     \begin{tabular}{|c|c|c|c|}
%         \hline
%         Sample & Uniform & Leverage & Sensitivity \\
%         \hline
%         50 & 0.5725 & 0.6903 & \textbf{0.5299}  \\
%         \hline
%         100 & 0.5093 & 0.6385 & \textbf{0.4379} \\
%         \hline
%         200 & 0.4687 & 0.5548 & \textbf{0.3231} \\
%         \hline
%         500 & 0.3777 & 0.3992 & \textbf{0.2173} \\
%         \hline
%         1000 & 0.2548 & 0.2318 & \textbf{0.1292} \\
%         \hline
%     \end{tabular} % &
%     \caption{All values are in order $10^{-3}$.}
%     \label{tab:syntheticTopicModeling}
% \end{table}

\noindent\textbf{Dataset:} We generated a dataset with 200k rows in $\mathbb{R}^{30}$. Each coordinate of row vector was uniformly generated entry between $0$ and $1$. Further each vector was normalized to have $\ell_{2}$ norm as 1. Hence we had a matrix of size $200k \times 30$ but we ensured that it had rank $12$. Furthermore 99.99\% of the rows in the matrix spanned only an 8-dimensional subspace in $\Re^{30}$ and its orthogonal $4$ dimensional subspace was spanned by the remaining 0.01\% of the rows. We simulated these rows to come in online fashion and applied the 3 sampling strategies. We generated 3-mode tensors $\mcal{\hat{T}}$ using the sampled rows and tensor $\mcal T$ sing the entire dataset.\\
\noindent\textbf{Uniform:} Here we sample rows uniformly at random from the dataset. It means that every row has a chance of getting sampled with a probability of $1/n$. Intuitively it is highly unlikely to pick rows from the subspace with few rows. Hence the required property might not be preserved for $\mathbf{x}$ coming from that particular row space. \\
% 
\noindent\textbf{Leverage:} Here we sample rows based on online leverage scores $c_{i} = \*a_{i}^{T}(\*A_{i}^{T}\*A_{i})^{-1}\*a_{i}$. We define a sampling probability for an incoming row $i$ as $p_{i} = c_{i}/(\sum_{j=1}^{i}c_{j})$. Rows with high leverage scores have higher chance of getting sampled. Though leverage score sampling preserved rank of the the data, but it is not known to preserve  higher order moments of the data. Intuitively this can be attributed to the fact that the sum of leverage scores is less than the sum of sensitivity scores.\\
%
\noindent\textbf{Sensitivity:} Sample rows of $\*A$ as per algorithm \ref{alg:onlineCoreset}.

The following tables compare the error $|\mathcal{T}(\*x,\*x,\*x) - \mathcal{\hat{T}}(\*x,\*x,\*x)|/(|\mathcal{T}(\*x,\*x,\*x)|)$ values  between three sampling schemes mentioned above. Here $\mathcal{T}(\*x,\*x,\*x) = \sum_{i=1}^{n}(\*a_{i}^T\*x)^3$. In table 1, $\*Q$ is set of right singular vectors of $\*A$ corresponding to the 5 smallest singular values. The table reports $|\sum_{\*x\in[\*Q]}\mathcal{T}(\*x,\*x,\*x) - \sum_{\*x\in[\*Q]}\mathcal{\hat{T}}(\*x,\*x,\*x)|/(|\sum_{\*x\in[\*Q]}\mathcal{T}(\*x,\*x,\*x)|)$. The table 2 reports for $\*x$ as the right singular vector of the smallest singular value of $\*A$. $\hat{\mathcal{T}}$ is the tensor we get from different sampling techniques. For each sampling technique and each sample size we ran $5$ random experiments and report the  mean of the experiments. Here as a result of the sampling techniques, the sample sizes mentioned are in expectation and not exact. After applying 2 sampled t-test on our expected sample size of $350$, the p-value for uniform and sensitivity score based sampling for the table 1 it is $0.0013$ and for leverage score and sensitivity score is $0.0793$. For the table 2 with same expected sample size the p-value for uniform and sensitivity score based sampling is $0.0103$ and for leverage score and sensitivity score is $0.0903$.

\begin{table}[htbp]
    \centering
    \begin{tabular}{|c|c|c|c|}
    \hline
    Sample & Uniform & Leverage & Sensitivity \\
    \hline
    200 & 1.0195 & \textbf{0.3641} & 0.4814  \\
    \hline
    250 & 1.0126 & 0.7021 & \textbf{0.4555} \\
    \hline
    300 & 1.0650 & \textbf{0.3697} & 0.4307 \\
    \hline
    350 & 1.0095 & 0.5355 & \textbf{0.3948} \\
    \hline
    \end{tabular}
    \caption{Error with $x \in [Q]$}
    \label{tab:query_set}
\end{table}
\begin{table}[htbp]
    \centering
\begin{tabular}{|c|c|c|c|}
        \hline
        Sample & Uniform & Leverage & Sensitivity \\
        \hline
        200 & 1.0000 & \textbf{0.5909} & 1.0437 \\
        \hline
        250 & 1.0000 & 0.6781 & \textbf{0.6737} \\
        \hline
        300 & 1.0000 & 0.9135 & \textbf{0.6598} \\
        \hline
        350 & 1.0000 & 0.7431 & \textbf{0.4575} \\
        \hline
    \end{tabular}\\
    \caption{Error with $x$ as right singular vector of the smallest singular value.}
    \label{tab:max_variance}
\end{table}
%%%%%%%%%%%%%%%%%%%%%%%%%%%%%%%%%%%%%%%%%%%%%%%%%%%%%%%%%%%%%%%%%%%%%%%%%%%%%%%
%%%%%%%%%%%%%%%%%%%%%%%%%%%%%%%%%%%%%%%%%%%%%%%%%%%%%%%%%%%%%%%%%%%%%%%%%%%%%%%


\end{document}

% Factorizing tensors has recently become an important optimization module in a number of machine learning pipelines, especially in latent variable models. We show how to do this efficiently in the streaming setting. Given a set of $n$ vectors, each in $R^d$, we present algorithms to 
% select a sublinear number of these vectors as coreset, while guaranteeing that the CP decomposition of the p-moment tensor of the coreset approximates the corresponding decomposition of the $p$-moment tensor computed from the full data. We introduce two novel algorithmic techiques: online filtering and kernelization. Using these two, we  
% present four algorithms that achieve different tradeoffs of coreset size, update time and working space, beating or matching various state of the art algorithms. In case of matrices (2-ordered tensor) our online row sampling algorithm guarantees $(1 \pm \epsilon)$ relative error spectral approximation. We show applications of our algorithms in learning Gaussian mixture models and topic modeling. 



% \begin{table*}[]
% \begin{tabular}{|l|l|l|l|}
% \hline
% Algorithm & Sample Size $\tilde{O}(\cdot)$            & Update time                     & Working space\\
% \hline\hline
% \mrwcb~\cite{dasgupta2009sampling} & $\tilde{O}(d^{p}k\epsilon^{-2})$  & $O(d^{5}p\log d)$ & $\tilde{O}(d^{p}k\epsilon^{-2})$ \\
% \hline
% \mrlw~\cite{cohen2015p} & $\tilde{O}(d^{p/2}k\epsilon^{-5})$  & $O(d^{p/2})$ & $\tilde{O}(d^{p/2}k\epsilon^{-5})$ \\
% \hline
% \mrFC \cite{clarkson2016fast} & $\tilde{O}(d^{p/2}k\epsilon^{-5})$  & $O(d^{p/2})$ & $\tilde{O}(d^{p/2}k\epsilon^{-5})$ \\
% \hline
% \hline
% \online$(p)$ (Theorem~\ref{thm:Online}) & $\tilde{O}(n^{1-2/p}dk\epsilon^{-2})$ & $O(d^{2})$ & $O(d^2)$ \\
% \hline
% \online$(p)$ + \mrlw (Theorem~\ref{thm:improvedStream-MR}) & $\tilde{O}(d^{p/2}k\epsilon^{-5})$ & $O(d^{2})$ amortized & $\tilde{O}(d^{p/2}k\epsilon^{-5})$
% %$O((1-2/p)^{10}d^{p/2}k(\log n)^{10}\epsilon^{-5})$
% %$O((1-2/p)^{11}d^{p/2}k(\log n)^{11}\epsilon^{-5})$
% \\
% \hline
% \kernel-\online$(2)$ (Theorem~\ref{thm:slowOnline}) & $\tilde{O}(d^{(p+1)/2}k\epsilon^{-2})$ & $O(d^{p})$ & $O(d^{p+1})$ \\
% \hline
% \online$(p)$-\kernel-\online$(2)$ (Theorem~\ref{thm:improvedOnlineCoreset}) & $\tilde{O}(d^{(p+1)/2}k\epsilon^{-2})$ & $O(d^{2})$ amortized & $O(d^{p+1})$ \\
% \hline 

% \end{tabular}
% \end{table*}
